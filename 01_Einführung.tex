\section{Einführung}
%\ifx\allfiles\undefined
\sethlcolor{yellow}
\subsection{\textbf{Aufgaben und Inhalt}}
Siehe \href{run:Bilder/ Referenzen/ Mitschrift-MWV-SS2021-Handout.pdf}{Messwertverarbeitung und Diagnosetechnik}.
\subsection{\textbf{Klassifikation der Signale {\itshape x(t)}}}
Einteilungskriterium:\par
%\hspace{1em} - nach der zeitlichen Quantisierung\par
%\hspace{2em} $\bullet$ Ein Signal x(t) ist \hl{zeit-kontinuierlich}, wenn die Amplitude zu jedem beliebigen Zeitpunkt veränderbar ist.\par
\hspace{1em} - Nach der zeitlichen Quantisierung\par
\begin{itemize}[topsep=2pt]
     \item
     Ein Signal \itshape x(t) ist \hl{zeit-kontinuierlich}, wenn die Amplitude zu jedem beliebigen Zeitpunkt veränderbar ist.
     \item
     Ein Signal \itshape x(t) ist \hl{zeit-diskret}, wenn die Amplitude nur zur diskreten Zeitpunkten veränderbar ist.
\end{itemize}

\hspace{1em} - Nach Quantisierung der Amplitude
\begin{itemize}[topsep=2pt]
     \item
     \hl{analoges Signal}: Signale, die kontinuierlich veränderliche Zeit und Amplituden haben.
     \item
     \hl{digitales Signal}: Signale, deren Amplitude nur auf endliche Anzahl diskreter Werte beschränkt, zeit-diskretes Signal.
\end{itemize}

\hspace{1em} - Reelle und komplexe Signale\par
\hspace{1em} - Nach der Reproduzierbarkeit\par
\begin{itemize}[topsep=2pt]
    \item 
    deterministische Signale
    \item
    zufällige Signale (stochastische Signale)
\end{itemize}

% Einfügen von "Klassifikation_Signale.pdf", die Bildformat kann *.jpg, *.png, *.pdf, *.eps, *.tiff usw.
\begin{figure}[H]
\centering
\includegraphics[width=105mm]{Bilder/Bilder_01/Klassifikation_Signale.pdf}
\caption{Klassifikation\ von\ Signalen}
\label{fig:env}
\end{figure}

%\clearpage
\hspace{1em} - Einteilung nach der Leistung/ Energie:\par

% Wrapfigure: 3 Zeile, links gelegt, Breite 12.5em
\begin{figure}[H]
    \centering
    \subfigbottomskip=10pt
    \subfigcapskip=-3pt
    \subfigure[Bsp.:Ohm'sches Gesetz]{
		\includegraphics[width= 50mm]{Bilder/Bilder_01/Widerstand.png}}
	\subfigure[Formeldarstellung]{
		\includegraphics[width=70mm]{Bilder/Bilder_01/Formel_1.png}}
	  \\
	\caption{Physikalische Analogie}
\end{figure}
\hspace{1em} Definition:\par
\hspace{2em} - \hl {Momentanleistung} eines Signals {\itshape x(t)}:\par

\begin{equation}
    P_{x} (t)=x^{2} (t) \cdots Autoleistung
\end{equation}
\begin{equation}
    P_{xy} (t)=x(t)\cdot y(t) \cdots Kreuzleistung
\end{equation}

\hspace{2em} - \hl {Die mittelere Leistung} eines Signals {\itshape x(t)}:\par
\begin{equation}
    \bar{P_{x}} =\lim_{T \to \infty} \frac{1}{2T} \int\limits_{-T}^{T} x^{2} (t)dt
\end{equation}

\hspace{2em} - \hl {Die Gesamtenergie} eines Signals {\itshape x(t)}:\par
\begin{equation}
    E_{x}=\int\limits_{-\infty }^{\infty} P_{x} (t)dt=\int\limits_{-\infty}^{\infty}x^{2} (t)dt
\end{equation}


\begin{itemize}[topsep=2pt]
     \item
     \hl{Leistungssignal}: Ein Signal {\itshape x(t)} ist ein Leistungssignal, wenn seine Energie bzw. sein Integral der Leistung über unendlichen Zeitbereich ebenfalls unendlich ist. \par
     \begin{figure}[h]
         \centering
         \subfigbottomskip=10pt
         \subfigcapskip=3pt
         \subfigure[x(t)=sin($\omega$ t)]{
             \includegraphics[width=0.3\linewidth]{Bilder/Bilder_01/Sinus_Funktion.png}
         }\hspace{2cm}
         \subfigure[Mathematische\ Beschreibung]{
             \includegraphics[width=0.3\linewidth]{Bilder/Bilder_01/Leistungssignal.png}
        }
     \caption{Leistungssignal}
     \end{figure}
     
     \item
     \hl{Energiesignal}: Ein Signal {\itshape x(t)} ist ein Energiesignal, wenn seine Gesamtenergie begrenzt ist: $E_{x}<\infty \to \bar{P_{x}}=0$\par
\end{itemize}
     
\begin{figure}[H]
    \centering
    \includegraphics[width=0.5\linewidth]{Bilder/Bilder_01/Sinus_abklingt.png}
    \caption{Abklingende Sinus-Funktion}
    \label{fig:Abklingende Sinus-Funktion}
\end{figure}

\hspace{2em} - Nach der Kausalität (kausales Signal):\par
\begin{equation}
    x(t)=\left\{
	\begin{aligned}
	0, \quad \mbox{für}\ x<0\\
	beliebig, \quad \mbox{für}\ x\ge 0
	\end{aligned}
	\right
	.
\end{equation}

\subsection{\textbf{Mathematische Grundlagen}}
\subsubsection{\textbf{Harmonische Funktion}}
\begin{figure}[h]
    \centering
    \subfigbottomskip=10pt
    \subfigcapskip=3pt
    \subfigure[Sinusfunktion]{
        \includegraphics[width=0.4\linewidth]{Bilder/Bilder_01/Math_Sinus.png}
    }
    \subfigure[Parameter]{
        \includegraphics[width=0.4\linewidth]{Bilder/Bilder_01/Param_Sinus.png}
    }
    \caption{Einfache harmonische Funktion}
    \label{fig:Einfache harmonische Funktion}
\end{figure}

\begin{figure}[H]
    \centering
    \subfigbottomskip=10pt
    \subfigcapskip=3pt
    \subfigure[sin($\omega$t)]{
        \includegraphics[width=0.3\linewidth]{Bilder/Bilder_01/sin(omegaxt).png}
    }
    \subfigure[sin($\omega$t-$\varphi$)]{
        \includegraphics[width=0.3\linewidth]{Bilder/Bilder_01/sin(omegaxt-phi).png}
    }
    \subfigure[sin($\omega$t+$\varphi$)]{
        \includegraphics[width=0.3\linewidth]{Bilder/Bilder_01/sin(omegaxt+phi).png}
    }
    \caption{Caption}
    \label{fig:my_label}
\end{figure}


\subsubsection{\textbf{Komplexe Zahlen}}

\begin{figure}[H]
    \centering
    \subfigbottomskip=10pt
    \subfigcapskip=3pt
    \subfigure[]{
        \includegraphics[align=c,width=0.4\linewidth]{Bilder/Bilder_01/reele_achse.png}
    }
    \subfigure[]{
        \includegraphics[align=c,width=0.5\linewidth]{Bilder/Bilder_01/komp_zahl.jpg}
    }
    \caption{Strahl für reelle sowie komplexe Zahlen}
    \label{fig:Komplexe Zahl}
\end{figure}

\begin{comment}
\begin{figure}[H]
    \centering
    \begin{minipage}[c]{.45\textwidth}
        \subfigure[Strahl für reele Zahlen]{
        \includegraphics[align=c, width=.9\linewidth]{Bilder/Bilder_01/reele_achse.png}}
    \end{minipage}
    \begin{minipage}[t]{.45\textwidth}
         \subfigure[Strahl für komplexe Zahlen]{
        \includegraphics[align=c, width=1\linewidth]{Bilder/Bilder_01/komp_zahl.jpg}
    }
    \end{minipage}
\end{figure}
\end{comment}


\begin{itemize}[topsep=2pt]
    \item 
    \underline{z} =a+jb, \quad j=$\sqrt{-1}$
    \item
    In Polarkoordinate:\par
    z=r \cdot $e^{j\varphi}=\sqrt{a^{2}+b^{2}} \cdot e^{j\varphi}$
    \item
    Trigonometrische Form:\par
    $z=\underbrace{r \cdot cos(\varphi)}_{\textcolor{blue}{a}}+j \cdot \underbrace{r \cdot sin(\varphi)}_{\textcolor{blue}{b}}=r \cdot (cos(\varphi)+j \cdot sin(\varphi))$
    \item[]
    \fbox{
        \parbox{0.95\textwidth}{
        \hl{Eulersche-Formel:}
        \begin{center}
             $e^{j\varphi}=cos\varphi+j \cdot sin\varphi$\par
             $cos\varphi=\frac{1}{2} \cdot (e^{j\varphi}+e^{-j\varphi}), \quad    sin\varphi=\frac{1}{2j} \cdot (e^{j\varphi}-e^{-j\varphi})$
        \end{center}
        }
    }
    \item
    Komplexe konjugierte Zahl:\par
    $\underline{z}=a+jb, \quad \underline{z^{*}}=a-jb$
    \item
    Rechenregel von Komplexer Zahl (Selbsstudium)
    \item
    Umwandlung zwischen verschiedenen Darstellungsformen der komplexen Zahlen:\par
    \begin{figure}[H]
        \centering
        \includegraphics[width=0.5\textwidth]{Bilder/Bilder_01/umwandlung_komp_zahl.png}
        \caption{Verschiedene Darstellungen komplexer Zahlen}
        \label{fig:Verschiedene Darstellungen komplexer Zahlen}
    \end{figure}
    \item[]
    \centerline{$\underline{z}=a+jb=r \cdot e^{j\varphi}$}
\end{itemize}


\subsubsection{\textbf{Komplexe harmonische Funktion}}
\begin{figure}[H]
    \centering
    \includegraphics[width=0.5\linewidth]{Bilder/Bilder_01/komp_harm_funk.png}
    \caption{Komplexe harmonische Funktion}
    \label{fig:Komplexe harmonische Funktion}
\end{figure}

\begin{equation}
    Ae^{j\omega t}=A \cdot cos(\omega t)+jA \cdot sin(\omega t)
\end{equation}

 \fbox{
        \parbox{0.95\textwidth}{
        \hl{Wiederholung: Eulersche-Formel:}
        \begin{center}
        $e^{j\varphi}=cos\varphi+j \cdot sin\varphi$\par
        $cos\varphi=\frac{1}{2} \cdot (e^{j\varphi}+e^{-j\varphi}), \quad sin\varphi=\frac{1}{2j} \cdot (e^{j\varphi}-e^{-j\varphi})$
        \end{center}
        }
    }



\subsubsection{\textbf{Überlagerung zwei Harmonischen gleicher Frequenz}}
\begin{equation}
    x(t)=x_{1}(t)+x_{2}(t)=A \cdot \cos(\omega_{0}t)+B \cdot \sin(\omega_{0}t)
\end{equation}

\fbox{
     \parbox{0.95\textwidth}{
        \hl{Additionstheorem aus Mathematik:}
        \begin{center}
        $sin(x+y)=sinx \cdot cosy+siny \cdot cosx$\\
        $cos(x+y)=cosx \cdot cosy-sinx \cdot siny$\\
        \end{center}
      }  
}

\begin{equation}
\begin{split}
     x(t)=& A \cdot \cos (\omega_{0}t)+B \cdot \sin(\omega_{0}t)\\
         =& \underbrace{\sqrt{A^{2}+B^{2}}}_{\textcolor{red}{C} }   \cdot \Bigl(\underbrace{\underbrace{\frac{A}{\sqrt{A^{2}+B^{2}}}}_{\textcolor{red}{\sin \psi} }}_{\textcolor{blue}{\cos \varphi} }   \cdot \cos(\omega_{0}t)+\underbrace{\underbrace{\frac{B}{\sqrt{A^{2}+B^{2}}}}_{\textcolor{red}{\cos\psi}}}_{\textcolor{blue}{\sin \varphi}}  \cdot \sin(\omega_{0}t )\Bigr)  \\
         =& \textcolor{red}{C \cdot \sin(\omega_{0}+\psi)}\\
         =& \textcolor{blue}{C \cdot \cos(\omega_{0}t+\varphi)}\\
     \psi=&\arctan (\dfrac{A}{B})\\
  \varphi=&\arctan (\dfrac{-B}{A})   
\end{split}
\end{equation}

Für die Bestimmung der Winkel $\psi$ bzw. $\varphi$ im Wertbereich $0-2\pi$ benutzt man die \hl{atan2} Funktion. In MatLab $atan2(y,x)$:\: \underline{$\psi$=atan2(A,B) \quad $\varphi$=atan2(-B,A)}.

\fbox{
     \parbox{0.95\textwidth}{
        \hl{Zusammenfassung:}\par
        $
        x(t)=A \cdot \cos(\omega_{0}t)+B \cdot \sin(\omega_{0}t)=C\sin(\omega_{0}+\varphi)=C \cdot \sin(\omega_{0}+\psi)\\
       \cdot C=\sqrt{A^{2}+B^{2}}\\
        \cos\varphi=\frac{A}{C}, \quad \sin\varphi=-\frac{B}{C}\\
        \sin\psi=\frac{A}{C}, \quad \cos\psi=\frac{B}{C}\\
        \varphi=atan2(-B, A), \quad \psi=atan2(A, B)
        $\\
        \textcolor{blue}{\hl{atan2 Funktion} liefert den Wertbereich ($-pi, pi$), Syntax in MatLab:\; atan2(y, x).}
        }
    }
    
\subsubsection{\textbf{Überlagerung zweier Harmonischen verschiedener Frequenzen}}

\begin{flalign*}
    & \  x_1=A \cdot \cos(\omega_1t)
    & \ & x_2=B \cdot \cos(\omega_2t)
    && \\
    & \ \omega_1=2\pi f_1
    & \ & \omega_2=2\pi f_2
    && \\
    & \ T_1=\frac{1}{f_1}
    & \ & T_2=\frac{1}{f_2}
\end{flalign*}

\begin{figure}[H]
    \centering
    \includegraphics[width=0.7\textwidth]{Bilder/Bilder_01/Ueber_zwei_versch_Freq.png}
    \caption{Überlagerung zweier Harmonischen verschiedener Frequenz}
    \label{fig:Ueber_zwei_versch_Freq}
\end{figure}

\fbox{
    \parbox{0.95\textwidth}{
    \begin{itemize}
        \item 
        \hl{Wichtig:} \\
        Die Summe von zwei Harmonischen verschiedener Frequenz ist \hl{nicht mehr harmonisch}.
        \item
        Die Summe zweier periodischen Funktion wird \hl{nur dann periodisch}, wenn \hl{das Frequenzverhältnis zweier Funktionen rational ist}.
        \item[]
        \hspace{4em}$\frac{\omega_2}{\omega_1}=\frac{f_2}{f_1}=\frac{T_1}{T_2}=\frac{m}{n}$, m, n - ganze Zahlen, teilerfremd
        \item
        Die Periodendauer der resultierenden Funktion {\itshape x(t)}:
        \item[]
        \hspace{4em}$T=kgV(T_1,T_2), \quad T=mT_2=nT_1, \quad f_0=ggT(f_1,f_2)$
        \item
        Wenn das Frequenzverhältnis nicht rational ist, dann ist die {\itshape x(t)} nicht periodisch.
        \item[]
        \hspace{4em}$\frac{T_1}{T_2}\approx \frac{m}{n}$, \quad für relativ große ganze Zahlen m,n $\rightarrow$ {\itshape x(t)} ist \hl{quasi-periodisch}. 
    \end{itemize}
    }
}

\textbf{Beispiel 1:}\\
{\itshape $x(t)=5\cos(4t)+4sin(10\pi t)$}\\
Frage: Ist {\itshape x(t)} periodisch?\\
\textcolor{blue}{Das Frequenzverhältnis $\dfrac{\omega_2}{\omega_1}=\dfrac{10\pi}{4}$} ist \textcolor{red}{irrational $\Rightarrow$ {\itshape x(t)} ist nicht periodisch.}

\textbf{Beispiel 2:}\\
{\itshape $y(t)=5\cos(10\pi t)+7sin(30\pi t+0.5)$}\\
Frage: Ist {\itshape y(t)} periodisch? Wenn ja, wie groß ist seine Periode?\\
\textcolor{blue}{Das Frequenzverhältnis $\dfrac{10\pi}{30\pi}=\dfrac{1}{3}$} ist \textcolor{red}{rational $\Rightarrow$ {\itshape y(t)} ist periodisch.} \\
$T_1=\dfrac{1}{5}, \quad T_2=\dfrac{2\pi}{\omega_2}=\dfrac{1}{15}, \quad T=kgV(T_1,T_2)=\dfrac{1}{5}$\\
$f_1=5, \quad f_2=15, \quad f=ggT(f_1,f_2)=5$\\



\subsubsection{\textbf{Spaltfunktion}}
Nicht normierte Form:  $si(t)=\dfrac{\sin t}{t}$\\
normierte Form:        $sinc(t)=\dfrac{\sin(\pi t)}{\pi t}$\\
\begin{equation}
    \mbox{Für t=0:}
    si(0)=\lim_{t \to 0} \dfrac{\sin t}{t}=\lim_{t \to 0}  \dfrac{\cos t}{1}=1 
\end{equation}

\begin{figure}[H]
    \centering
    \includegraphics[width=0.8\textwidth]{Bilder/Bilder_01/si(t)_Überlagerung.png}
    \caption{Entstehung der Spaltfunktion Si(t)}
    \label{fig:si(t)_Überlagerung}
\end{figure}

\fbox{
    \parbox{0.95\textwidth}{
    \hl{Eigenschaften:}
    \begin{itemize}
        \item [a)]
        $si(t)$ ist eine \uuline{gerade} Funktion. $si(t)=si(-t)$
        \item [b)]
        $si(0)=1$
        \item [c)]
        Nullstellen von $si(t)$: $\;$ $\pm \pi, \pm 2\pi, \pm 3\pi, \cdots$
    \end{itemize}
    }
}


%%% wrapfigure: Bild rechts gestellt, während Sätze andererseitig
%%% Param: [4] vertikal 4 Zeilen, {r} Bild recht liegend, {12em} Breite

\begin{comment}
\begin{itemize}
    \item 
    \begin{wrapfigure}{r}{6cm}
        \centering
        \includegraphics[width=1\linewidth]{Bilder/Bilder_01/gerade_Funktion.png}
        \caption{Gerade Funktion}
        \label{fig:gerade}
    \end{wrapfigure}
    Eine Funktion $x(t)$ ist gerade, wenn es gilt:
    $x(t)=x(-t)$ $\Rightarrow$ achsensymmetrisch
    \item 
    \begin{wrapfigure}{r}{6cm}
        \centering
        \includegraphics[width=1\linewidth]{Bilder/Bilder_01/ungerade_Funktion.png}
        \caption{Ungerade Funktion}
        \label{fig:ungerade}
    \end{wrapfigure}
    Eine Funktion $x(t)$ ist ungerade, wenn es gilt:
    $x(t)=-x(-t)$ $\Rightarrow$ punktsymmetrisch
\end{itemize}
\end{comment}

\vspace{1.5cm}
\fbox{
   \parbox{.95\textwidth}{
       \begin{minipage}{.45\linewidth}
       \hl{Definition:}
            \begin{itemize}
                \item Eine Funktion $x(t)$ ist gerade, wenn es gilt:$x(t)=x(-t)$ $\Rightarrow$ achsensymmetrisch
                \item Eine Funktion $x(t)$ ist ungerade, wenn es gilt:$x(t)=-x(-t)$ $\Rightarrow$ punktsymmetrisch
            \end{itemize}
        \end{minipage}\vspace{0.5cm}
        \begin{minipage}{.45\linewidth}
            \hspace{0.6cm}
            \includegraphics[scale=0.15]{Bilder/Bilder_01/gerade_Funktion.png}
            %\caption{Gerade Funktion}
            \label{fig:gerade}
            \hspace{0.8cm}
            \includegraphics[scale=0.15]{Bilder/Bilder_01/ungerade_Funktion.png}
            %\caption{Ungerade Funktion}
            \label{fig:ungerade}
        \end{minipage}
   }
}




\subsubsection{\textbf{DIRAC-Impuls}}

%%% \{minipage} ist nur innerhalb eines Containers verwendbar
\begin{figure}[H]
    \begin{minipage}{.45\linewidth}
        \centering
        \includegraphics[scale=0.2]{Bilder/Bilder_01/Dirac-Impuls.png}
    \end{minipage}
    \begin{minipage}{.45\linewidth}
            \begin{equation}
                r_\tau (t)=\left\{
                \begin{array}{cl}
                    \dfrac{1}{\tau}, &  \ \mbox{\itshape{für}} -\dfrac{\tau}{2}\leq t\leq \dfrac{\tau}{2} \\
                    0,  &  \ sonst \\
                \end{array} \right.
                \nonumber
            \end{equation}
        \end{minipage}
\end{figure}

Die DIRAC-Impuls ist definiert als:\\
\begin{equation}
    \delta (t)=\lim_{\tau \to 0}r_\tau (t) 
\end{equation}

\fbox{
    \hl{Definition:}\\
    \begin{minipage}{.45\linewidth}
        \begin{figure}[H]
            \centering
            \includegraphics[scale=0.5]{Bilder/Bilder_01/Dirac.png}
            \caption{Dirac-Impuls}
            \label{fig:dirac}
        \end{figure}
    \end{minipage}
    \begin{minipage}{.45\linewidth}
        \begin{equation}
            \delta = \left\{
            \begin{array}{cc}
                 0, & \ \mbox{für} t\neq 0 \\
                 \neq 0, & \ \mbox{für} t=0 \\
                 \end{array} \right.
                 \nonumber
        \end{equation}
    \end{minipage}
}

\vspace{0.5cm}
\hl{Eigenschaften:}
\begin{enumerate}
    \item 
    $\delta (t)$ ist eine gerade Funktion, $\delta (t)=\delta (-t)$
    \item
    $\int\limits_{-\infty}^{\infty} \delta(t)dt=1$
    \item
    \hspace{1.77cm} 
    \begin{figure}[H]
        \centering
        \begin{minipage}{.45\textwidth}
            \centering
            \includegraphics[scale=0.6]{Bilder/Bilder_01/Eigen_delta_01.png}
        \end{minipage}
        \begin{minipage}{.45\textwidth}
            \begin{equation}
                \begin{split}
                    x_s(t)&\approx \cdots r_{\Delta t} \cdot (t-k\Delta t) \cdot \Delta t \cdot x(k\Delta t)\cdots\\
                    &=\sum_{k=-\infty}^{\infty} r_{\Delta t} \cdot (t-\textcolor{blue}{\underbrace{k\Delta t}_\tau}) \cdot x(\textcolor{blue}{\underbrace{k\Delta t}_\tau}) \cdot \textcolor{blue}{\underbrace{\Delta t}_{d\tau}} \\
                    & k \cdot \Delta t \rightarrow \tau \\
                    & \Delta t \rightarrow d\tau 
                \end{split}
                \nonumber
            \end{equation}
        \end{minipage}
    \end{figure}
\end{enumerate}

Bei der Eigenschaft \textbf{(3)}, wenn $\Delta t$ von Null tendiert, gilt:
\begin{equation}
    \begin{split}
      x(t)=&\lim_{\Delta t \to 0}x_s(t)\\
          =&\lim_{\Delta t \to 0}\sum_{k=-\infty}^{\infty} r_{\Delta t} \cdot (t-k\Delta t) \cdot x(k\Delta t) \cdot \Delta t \\
          =&\int\limits_{-\infty}^{\infty}\delta (t-\tau)x(\tau)d\tau  
    \end{split}
    \nonumber
\end{equation}
Daraus erhält man die Gleichung:\\
\begin{equation}
    \boxed{x(t)=\int\limits_{-\infty}^{\infty}x(\tau) \cdot \delta (t-\tau)d\tau}
\end{equation}
\\Wobei, $\delta (t-\tau)$ ist der Dirac-Impuls, der als seine Verschiebung entlang der t-Achse nach rechts um eine Stecke von t angesehen werden kann.\\

Daraus ergeben sich zwei Sonderfälle der Gleichung \; (11).

\begin{equation}
    \begin{split}
        &x(t)\delta (t)=x(0)\delta (t)\\
        &\int\limits_{-\infty}^{\infty}x(t)\delta (t)dt=x(0)\\
        &\int\limits_{-\infty}^{\infty}x(t)\delta (t-t_0)dt=x(t_0)
    \end{split}
    \nonumber
\end{equation}


\subsubsection{Faltungsprodukt}

\begin{figure}[H]
    \centering
    \includegraphics[scale=0.65]{Bilder/Bilder_01/Faltungsprodukt.png}
    \caption{LTI-System}
    \label{fig:lti}
\end{figure}



\begin{figure}[H]
    \begin{minipage}{.45\linewidth}
        \centering
        \hspace{15 pt}
        \includegraphics[scale=0.7]{Bilder/Bilder_01/LTI-system.png}
        \nonumber
    \end{minipage}
    \begin{minipage}{.45\linewidth}
            \begin{equation}
               y(t)=T_r\{x(t)\}
            \nonumber
            \end{equation}
        \end{minipage}
\end{figure}

\begin{equation}
    \begin{split}
    y(t)=&T_r\left \{\int\limits_{-\infty }^{\infty }x(\tau ) \cdot \delta (t-\tau )d\tau \right \} \\
        =&\int\limits_{-\infty }^{\infty } T_r\left \{x(\tau ) \cdot \delta (t-\tau )d\tau \right \}\\
        =&\int\limits_{-\infty }^{\infty }x(\tau ) \underbrace{T_r\ \cdot {\delta (t-\tau )\}}_{\textcolor{blue}{Impulsantwort\; h(t),\; Gewichtfunktion}} d\tau 
\end{split}
\end{equation}

\fbox{
    \parbox{0.95\textwidth}{
        \hl{Faltungsprodukt:}
        \begin{equation}
            y(t)=\int\limits_{-\infty }^{\infty }x(\tau ) \cdot h(t-\tau )d\tau =x(t)\ast  h(t)
        \end{equation}
        Sonderfall:\\
        \hspace{1.77cm} Wenn $x(t)$ und $y(t)$ \uline{kausal} sind, d.h. sie sind für t<0 Null, denn gilt:\\
        \begin{equation}
            y(t)=x(t)\ast  h(t)=\int\limits_{0}^{\infty }x(\tau )h(t-\tau )d\tau
        \end{equation}
    }
}
\textbf{Eigenschaften:}
\begin{enumerate}
    \item 
    $x*h=h*x$
    \item
    $(x_1+x_2)*h=x_1*h+x_2*h$
    \item
    $(x*g)*h=x*(g*h)$
    \item
    \uuline{$x(t)*\delta(t-t_0)=x(t-t_0)$}
\end{enumerate}

\textbf{Grafische Darstellung des Faltungsvorgangs im Zeitbereich}:
\begin{figure}[H]
    \centering
    \begin{minipage}{.3\textwidth}
        \includegraphics[scale=0.3]{Bilder/Bilder_01/x0(t).png}
        \nonumber
    \end{minipage}
    \begin{minipage}{.03\textwidth}
            $\ast$
    \end{minipage}
    \begin{minipage}{.3\textwidth}
             \includegraphics[scale=0.3]{Bilder/Bilder_01/t0.png}
        \nonumber
    \end{minipage}
    \begin{minipage}{.03\textwidth}
            =
    \end{minipage}
    \begin{minipage}{.3\textwidth}
        \includegraphics[scale=0.3]{Bilder/Bilder_01/Faltungsprodukt_graphisch.png}
        \nonumber
    \end{minipage}
    \caption{Grafische Darstellung des Faltungsvorgangs}
    \label{fig:Faltung im Zeitbereich}
\end{figure}

Grafische Methode zur Berechnung der Faltung:\\
\begin{equation}
    y(t)=x(t)\ast h(t)=\int_{-\infty}^{\infty}x(\tau)h(t-\tau)d\tau
\end{equation}

\begin{figure}[H]
    \centering
    \begin{minipage}{.45\textwidth}
        \includegraphics[scale=0.5]{Bilder/Bilder_01/Faltung_grafisch.png}
    \end{minipage}
    \begin{minipage}{.45\textwidth}
        \begin{enumerate}
            \item[]
            \textbf{Schritte:}\\
            \item 
            Variable $\,$ t $\rightarrow$\ $\tau$. Skizze von x(t), $\tau(t)$.
            \item
            achsensymmetrische Spiegelung von $h(\tau) \rightarrow h(-\tau)$
            \item
            Verschiebung des gespiegelten Signals $\,$ $h(\tau)$ $\,$ von eimen Wert $t_1$\\
            $\Rightarrow$ $h(t_1-\tau)$
            \item
            Berechnung des Produktes der zwei Verläufe $x(\tau)h(t_1-\tau)$
            \item
            Bestimmung der Fläche unter dem Produkt der beiden Signalen, diese Fläche $\,$=$\,$ das Faltungsprodukt zum Zeitpunkt $t_1$.
            \item
            Wiederholung der Schritte \ding{194} bis \ding{196} für einen anderen t-Wert.
        \end{enumerate}
    \end{minipage}
    \caption{Schritte zur Berechnung einer Faltungsoperation}
\end{figure}

\begin{figure}[H]
    \centering
    \includegraphics[scale=0.4]{Bilder/Bilder_01/Zusammenfassung des Faltungsprozesses.png}
    \caption{Zusammenfassung des Faltungsprozesses}
    \label{fig:Zusammenfassung Faltung}
\end{figure}

\newpage