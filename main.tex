%%%%%%%%%%%%%%%%%%%%%%%%%%%%%%%%%%%%%%%%%%%%%%%%%%%%%%%%%%%%%%%%%%%%%%%%%%%%%%%%%
%%% TU Dresden - Stiftungsprofessur für Baumaschinen						  %%%
%%% Latex-Vorlage für Studien- und Abschlussarbeiten						  %%%
%%% Version: 1.7															  %%%
%%% Erstellt von: Robert Zickler											  %%%
%%% Verbesserungen, Fragen, Hinweise?: eMail an: robert.zickler@tu-dresden.de %%%
%%%%%%%%%%%%%%%%%%%%%%%%%%%%%%%%%%%%%%%%%%%%%%%%%%%%%%%%%%%%%%%%%%%%%%%%%%%%%%%%%

%Die Masterdatei ist die Datei, die von Latex übersetzt wird, sie enthält die Präambel in der die Formatierung festgelegt wird und zusätzliche Pakete geladen und konfiguriert werden. Der Name kann entsprechend geändert werden: z. B. 00Masterdatei.tex -> Diplomarbeit_Nachname.tex
%Dabei muss beachtet werden, dass bei jeder Dateiumbenennung, die kompilierten Zusatzdateien (*.lof, *.out, ...) im Ordner gelöscht werden müssen. Achtung: TEX- und BIB-Datei nicht löschen und öfters sichern ;) (z. B. mit GitHub o. Ä.)

\documentclass[11pt, twoside, numbers=noenddot, bibliography=totoc, listof=entryprefix, draft=false]{scrartcl}	%https://ctan.space-pro.be/tex-archive/macros/latex/contrib/koma-script/doc/scrguide.pdf
% fleqn					Gleichungne linksbündig 
% totoc					Quellen in das Inhaltsverzeichnis
% twoside 				Doppelseitiges Dokument (Duplex) (Ref: oneside)
% numbers=noenddot 		Kein Punkt hinter der Letzzten Kapitelnummer (z. B. 1.2.3. -> 1.2.3)
% listof=entryprefix	Tabellen- und Abbildungsverzeichnis mit Präfix "Tab." bzw. "Abb."
% draft					Entwurf (Bilder und PDF werden nicht eingebunden und durch Platzhalter ersetzt, Quellen werden nicht zugeordnet => schnelleres Übersetzen)

%\usepackage{xcolor}
% Bitte folgenden Absatz ausfüllen
% Einstellungen für die Meta-Daten der PDF-Datei. Einsehbar z. B. mit Adobe Reader: [Datei] -> [Eigenschaften...] -> [Beschriebung]
\usepackage[pdfauthor={Xueyong Lu},																	% <<<<<<<<=============
            pdftitle={00Masterdatei},														% <<<<<<<<=============
            pdfsubject={Art der Arbeit (Studienarbeit)},						% <<<<<<<<=============
            pdfkeywords={Wahlweise ein paar Stichworte (z. B. Latex, Vorlage, Beispiele, Baumaschine, ...}	% <<<<<<<<=============
            %pdfborderstyle={/S/U/W 1}, % underline links instead of boxes
            %linkbordercolor=blue,      % color of internal links
            %citebordercolor=blue,      % color of links to bibliography
            %filebordercolor=blue,      % color of file links
            %urlbordercolor=cyan        % color of external links
            ]{hyperref}		


%\usepackage{showframe} 				% Aktivieren (Auskommentieren) um sich die einzelnen Rahmen im Dokument anzeigen zu lassen


%%% Seiteneinstellungen (Seitenränder, Kopf und Fußzeile)
\usepackage[
	paper	= a4paper, 					% A4-Papierformat
	inner	= 3cm, 						% Da Doppelseitig gedruck: Abstand nach innen (zur Info die Alternative bei Einseitendruck: left)
	outer	= 2cm, 						% Da Doppelseitig gedruck: Abstand nach außen (zur Info die Alternative bei Einseitendruck: right)
	bottom	= 2.5cm,					% Seitenrand nach unten
	top		= 2.5cm,					% Seitenrand nach oben
	bindingoffset = 0mm					% Innerer Seitenrand ohne zusätzlichem Offset für die Bindung	
	]{geometry}				

% Folgendes Paket erzeugt fortlaufend eine Warnung. Diese kann aber ignoriert werden. Paket solle evtl. zukünftig durch das "scrlayer-scrpage"-Paket erstetzt werden
\usepackage{fancyhdr}					% Paket zur einfacheren Manipulation von Kopf- und Fußzeile 
\pagestyle{fancy} 						% http://www.mrunix.de/forums/showthread.php?65468-Section-Name-ohne-Nummer-in-die-Kopfzeile
\fancyhf{}								% Voreingestellte Kopf- und Fußzeile löschen
\fancyfoot[EL,OR]{\thepage}				% Seitenzahl am äußeren Rand des Seitenfußes
\renewcommand{\footrulewidth}{0.4pt}
\newcommand{\headerOn}{\fancyhead[EL,OR]{\nouppercase{\footnotesize\leftmark}}}	% eigener Befehl um die Kopfzeile zu deaktivieren
\newcommand{\headerOff}{\fancyhead[EL,OR]{\color{white}\leftmark}} 				% eigener Befehl um die Kopfzeile zu aktivieren
%%%%%%%%%%%%%%%%%%%%%%%%


%%% Unkategorisierte Pakete
\usepackage[colorlinks=true,allcolors=blue]{hyperref}
                                        % Bibliothek für Hypertext-Link
\usepackage{graphbox}                   % mit subfigure[align=c, scale=0.5]{*.png}
%\usepackage{pifont}
\usepackage{amsmath}
\numberwithin{equation}{section}        % Formel nach Kapiteln neu beziffern
\numberwithin{figure}{section}          % Abbildung nach Kapiteln neu beziffern
\usepackage{amssymb}
\usepackage{pifont}
\usepackage{wrapfig}
\usepackage{subfigure}
\usepackage{float}                      % Zugriff auf eine andere Datei in derselben Pfad
\usepackage{ulem}
\usepackage{ctex}
\usepackage{color,soul}                 % Unterstreichen und Hervorheben von Texten
\usepackage{verbatim}                   % Mehrzeilige Kommentare
\usepackage{multirow}					% Tabelle zwei Zeilen verbinden
\usepackage{xltabular}					% Tabelle bis an den Seitenrand bei "X" z. B.: \begin{tabularx}{\linewidth}{|l|X|}
\usepackage{booktabs}					% Zum Erstellen von Tabellen mit booktabs	\toprule, \midrule \bottomrule
\usepackage{longtable} 					% Für Tabellen über mehrere Seiten z. B. Formelverzeichnis
\usepackage{pbox}						% Zeilenumbrüche in Tabelle https://tex.stackexchange.com/questions/2441/how-to-add-a-forced-line-break-inside-a-table-cell/19678
\usepackage{array}						% Zum Erstellen von Arrays
\usepackage{mathrsfs}					% Von geogebra. Tipp: in Geogebra kann eine Darstellung direkt nach Latex exportiert werden
\usepackage{pdfpages} 					% Zum Einbinden einer PDF-Seite https://stackoverflow.com/questions/2739159/inserting-a-pdf-file-in-latex
\usepackage{rotating}					% Ermöglicht das Drehen einer oder mehrerer Seiten (Querformat). Kopf und Fußzeile bleiben davon unberührt (\begin(sidewaysfigure))
\usepackage{graphicx}					% Zum Einbinden von Bildern
\usepackage{blindtext}					% Zum Erzeugen von Blindtexten. Wird nur für diese Vorlage verwendet
%%%%%%%%%%%%%%%%%%%%%%%%


%%% Schriftarteinstellungen
\usepackage{fontspec}					% Für die Schriftauswahl
\setmainfont{OpenSans-Regular.ttf}[		% Open Sans (CD der TU Dresden) als Schriftart festlegen
	Path = Schriftarten/,
	BoldFont = OpenSans-Bold.ttf ,
	ItalicFont = OpenSans-Italic.ttf ,
	BoldItalicFont = OpenSans-BoldItalic.ttf ]
\urlstyle{same}							% URL mit selber Schriftart wie Text
\usepackage[mathrm=sym]{unicode-math}	% https://github.com/firamath/firamath Matheschriftart im Open Sans Style
\setmathfont{Fira Math}					% Schriftart für mathematische Ausdrücke
\usepackage{latexsym}					% Zum Erzeugen diverser Symbole und Zeichen
\usepackage[ngerman]{babel}             % Rechtschreibung "n" steht für neue dt. Rechtschreibung
\usepackage{csquotes}					% Darstellen der Zitate entsprechend der gewählten Rechtschreibregeln
\setlength{\parindent}{0pt} 			% Kein Einrücken nach Textumbruch
%\usepackage[onehalfspacing]{setspace}	% 1,5 facher Zeilenabstand. Nicht auskommentieren nur zur Referenz. Korrekte Einstellung wird im Folgenden definiert
%%%%%%%%%%%%%%%%%%%%%%%%


%%%% 1,5 facher Zeilenabstand entsprechend Word
\usepackage{setspace}
\makeatletter
\newcommand{\MSonehalfspacing}{%
  \setstretch{1.44}%  default
  \ifcase \@ptsize \relax % 10pt
    \setstretch {1.448}%
  \or % 11pt
    \setstretch {1.399}%
  \or % 12pt
    \setstretch {1.433}%
  \fi
}
\newcommand{\MSdoublespacing}{%
  \setstretch {1.92}%  default
  \ifcase \@ptsize \relax % 10pt
    \setstretch {1.936}%
  \or % 11pt
    \setstretch {1.866}%
  \or % 12pt
    \setstretch {1.902}%
  \fi
}
\makeatother
\MSonehalfspacing
%%%%%%%%%%%%%%%%%%%%%%%%


%%% Tabellenüber- und Abbildungsunterschriften
\renewcaptionname{ngerman}{\figurename}{Abb.} 	% "Abbildung" und "Tabelle" in "Abb." und "Tab." umbenennen
\renewcaptionname{ngerman}{\tablename}{Tab.}
\usepackage[
	font=small,
	labelfont=bf,
    justification=RaggedRight,	% Bildunterschrift linksbündig
    singlelinecheck=false		% Bildunterschrift linksbündig
	]{caption} 	% Bildunterschriften Schritarteigenschaften
	
% Abbildungsunterschriften mittg anlegen	
\usepackage[justification=centering]{caption}	

\def\allfiles{}

\begin{comment}
\usepackage[				% Bei mehreren Bildern in einer Abbiildung (subcaption). Darstellungseigenschaften
	font=small,				% Kleine Schriftart
	list=on, 				% Aufnehmen im LoF (List of Figures)
	listformat=simple,		% Darstellung im LoF
	labelfont=bf, 			% Fett schreiben
	labelformat=brace, 		% a) Klammer nach dem Bezeichner a), b),...
	position=bottom			% Label unterhalb
	]{subcaption}
\BeforeStartingTOC[lof]{\renewcommand*\autodot{:}}	% ListOfFigures mir Doppelpunkt z. B.: Tab 3: Beschriftung .... Nummer
\BeforeStartingTOC[lot]{\renewcommand*\autodot{:}}	%
\end{comment}
%%%%%%%%%%%%%%%%%%%%%%%%

%%% Einstellung der vertikalen Abstände zwischen Gleichungen
\begin{comment}
\makeatletter
\renewcommand\normalsize{%
\@setfontsize\normalsize\@xpt\@xiipt
\abovedisplayskip 5\p@ \@plus2\p@ \@minus5\p@
\abovedisplayshortskip \z@ \@plus5\p@
\belowdisplayshortskip 8\p@ \@plus5\p@ \@minus5\p@
\belowdisplayskip \abovedisplayskip
\let\@listi\@listI}
\makeatother
\end{comment}


%%% Einstellungen der Überschriften (Schriftgröße, Großbuchstaben, etc.
\addtokomafont{section}{\fontsize{18}{18} \selectfont\MakeUppercase}		% Schriftgröße Überschrift 1.
\addtokomafont{subsection}{\fontsize{14}{14} \selectfont\MakeUppercase}		% Schriftgröße Überschrift 1.1
\addtokomafont{subsubsection}{\fontsize{12}{12} \selectfont}				% Schriftgröße Überschrift 1.1.1
\addtokomafont{paragraph}{\fontsize{11}{11} \selectfont}					% Schriftgröße Überschrift 1.1.1.1
\setcounter{secnumdepth}{4}													% Numerierungstiefe bis einschließlich 4. Ebene z. B. 1.2.3.4
\setcounter{tocdepth}{3}													% Nur bis zur dritten ebene ins Inhaltsverzeichnis (default)
\DeclareNewSectionCommand[													% https://tex.stackexchange.com/questions/377032/newcommand-for-subsubsubsection
  style=section,
  level=4,
  beforeskip=-3.25ex plus -1ex minus -.2ex,
  afterskip=1.5ex plus .2ex,
  counterwithin=subsubsection
]{subsubsubsection}															% Eigener Befehl für die 4. Gliederungsebene \subsubsubcation | \paragraph{•} bitte nicht verwenden
%%%%%%%%%%%%%%%%%%%%%%%%


%%% SI-Einheiten http://texwelt.de/wissen/fragen/2588/wie-schreibe-ich-zahlen-mit-einheiten-richtig
\usepackage{siunitx}   							% SI-Einheiten
\sisetup{										% Handbuch: https://mirror.informatik.hs-fulda.de/tex-archive/macros/latex/contrib/siunitx/siunitx.pdf
  	locale = DE ,								% Deutsche Schreibweise
  	per-mode = fraction,						% Bruch-Schreibweise (zur Info die alternativen: reciprocal | symbol)
  	detect-all = true,							% Alle berücksichtigen
	detect-inline-family = math}				% Mathe-Schriftart bei Verwendung von SI-Einheiten im Text
% Eine Währung ist zwar keine SI-Einheit, für die Darstellung kann die Umgebung dennoch genutzt werden
\DeclareSIUnit{\sieuro}{\mbox{\euro}}			% Euro-Symbol hinzufügen
\DeclareSIUnit{\sidollar}{\mbox{\$}}			% Dollar-Symbol hinzufügen
\DeclareSIUnit{\siyuan}{\mbox{\textyen}}		% Yuan-Symbol hinzufügen
\DeclareSIUnit{\sipound}{\mbox{\pounds}}		% Pfund-Symbol hinzufügen
%%%%%%%%%%%%%%%%%%%%%%%%


%%% Gleichungen
% \numberwithin{equation}{section}				% Nicht verwenden, nur zur Refferenz. Gleichungen nach Kapitel nummerieren https://golatex.de/gleichungen-nummerieren-t3753.html 
\DeclareMathOperator{\arctantwo}{arctan2}		% darstellen von arctan2 mit dem Befehl \arctan2
%%%%%%%%%%%%%%%%%%%%%%%%


%%% Quellcode und Algorithmen
% https://ftp.rrzn.uni-hannover.de/pub/mirror/tex-archive/macros/latex/contrib/listings/listings.pdf
\usepackage{listings}										% Paket zum einbinden von Quellcode 
\renewcommand{\lstlistlistingname}{Quellcodeverzeichnis}	% Überschrift im Verzeichnis
\newcaptionname{ngerman}{\lstlistingname}{Quelltext}		% Beschriftung von Quelltextabbildungen
\lstset{
	frame=single,							% Horizontale Linie oben und unten
	numbers=left, %eller none
	xleftmargin=2em,						% Zeilennummer in den Rahmen
	framexleftmargin=1.5em,					% Zeilennummer in den Rahmen
	numberstyle=\tiny\color{darkgray},
	breaklines=true,
	breakatwhitespace=true,
	showstringspaces=false,					% Lehreichen ohne Markierung
	commentstyle=\color{darkgray},			% Farbe der Kommentare
	tabsize=3,
	escapeinside={(*@}{@*)},				% Definition des Befehles zum erzeugen eines Zeilenverweises
	% showtabs=true,tab=\rightarrowfill		% Durch einkommentieren lassen sich die Tabulatorzeichen anzeigen. In der finalen Datei bitte auskommentieren
}
\usepackage[
	german,
	linesnumbered,
	lined,
	boxed,
	ruled,
	commentsnumbered
	]{algorithm2e}							% Paket zum Erzeugen von Algorithmen
\SetNlSty{bfseries}{\color{darkgray}}{}		% Farbe Zeilennummerierung

%%%%%%%%%%%%%%%%%%%%%%%%

%%% Stichpunkte
\usepackage{enumitem}							% Handbuch: http://mirror.physik-pool.tu-berlin.de/pub/CTAN/macros/latex/contrib/enumitem/enumitem.pdf
\setitemize{itemsep=-0.15cm}					% Zeilenabstand bei Stichpunkten
\newlist{titemize}{itemize}{1}					% Neue Listenumgebung für Tabellen
\setlist[titemize]{nosep, label=\textbullet, leftmargin=*, after=\strut}
%%%%%%%%%%%%%%%%%%%%%%%%


%%% Zitate und Quellen
\usepackage[									% Handbuch: https://ftp.mpi-inf.mpg.de/pub/tex/mirror/ftp.dante.de/pub/tex/info/translations/biblatex/de/biblatex-de-Benutzerhandbuch.pdf
	style=numeric-comp,								% Zitierstile
	sorting = nty,									% Sortierungsschema Siehe Handbuch 3.1.2.1
	backend = biber,								% Biber, das Standardbackend von BibLaTeX (default)
	firstinits=true									% Nur den ersten Buchstaben des Vornamen anzeigen
	]{biblatex}										% natbib=true,  authoryear-ibid
\addbibresource{9999Quellen.bib}				% Datei in der alle Quellen aufgeführt werden
\DeclareNameAlias{default}{family-given}				%Familienname dann Vorname
\DefineBibliographyStrings{ngerman}{
   andothers = {{et\,al\adddot}},
}														% "u. A." in "et al."
\renewcommand*{\multinamedelim}{\addsemicolon\space}	% Trenner zwischen den Namen ein Semikolon
\renewcommand*{\labelnamepunct}{\addcolon\space}		% Doppelpunkt nach dem letzten Namen
\renewcommand*{\finentrypunct}{\addspace}				% Kein Punkt am Ende des Literaturverzeichnis
\renewcommand*{\prenotedelim}{\addnbspace}				% Kein Zeilenumbruch im Verweis https://golatex.de/autocite-kein-zeilenumbruch-t18069.html
\renewcommand*{\postnotedelim}{\addcomma\addnbspace}	% Komma statt Punkt
\renewcommand*{\multicitedelim}{\addcomma\addnbspace}	% Komma statt Punkt
\renewcommand*{\extpostnotedelim}{\addnbspace}			% für Abstände 
\renewcommand*{\volcitedelim}{\addcomma\addnbspace}		% für Abstände 
%%%%%%%%%%%%%%%%%%%%%%%%

%%% tikz - Umfangreiche Bibliothek für diverse Darstellungen, Grafen, ect.
\usepackage{tikz}
\usetikzlibrary{calc, positioning, shapes.geometric, arrows.meta, quotes, angles, patterns,decorations.pathmorphing,decorations.markings}	%je nach Bedarf beliebig anpassen
% weitere Einstellungen für die tikz-Erstellung von z. B. Ablaufplänen
\tikzset{
>={Triangle[]},
base/.style={draw, align=center, minimum height=1.75cm, minimum width=2cm},
    node distance=1.75cm,
    startstop/.style={rectangle, rounded corners = 3mm, minimum width=3cm, minimum height=1cm,text centered, draw=black, fill = black!20},
    process/.style={rectangle, minimum width=3cm, minimum height=1cm, text centered, draw=black},
    io/.style={trapezium, trapezium left angle=70, trapezium right angle=110, minimum width=3cm, minimum height=1cm, text centered, draw=black, fill=blue!30},
    decision/.style={base, diamond, aspect=2,  text width=10em, inner sep=-1pt},
    subroutine/.style = {rectangle, minimum width=3cm, minimum height=1cm, text centered, draw=black,  path picture={\draw([xshift=1mm]path picture bounding box.north west)--([xshift=1mm]path picture bounding box.south west)([xshift=-1mm]path picture bounding box.north east)--([xshift=-1mm]path picture bounding box.south east);}},
    }
\tikzset{										% zwei Farben in Pfeil
bicolor/.style 2 args={
  dashed,dash pattern=on 10pt off 10pt,->,#1,
  postaction={draw,dashed,dash pattern=on 10pt off 10pt,->,#2,dash phase=10pt}
  },
}
%%%%%%%%%%%%%%%%%%%%%%%%

%%% Graphen
\usepackage{pgfplots}   	% Graphen zeichnen
\pgfplotsset{compat=newest} % Legende unter dem Graph
\usepgfplotslibrary{units} 	% eingabe von Einheiten in pfgplots
\usetikzlibrary{arrows} 	% Pfeilspitzen
\pgfplotsset{				% Darstellung von Graphen festlegen (kann in den einzelnen Graphen separat angepasst werden)
	major grid style={dashed,gray!50},
    minor grid style={dashed,gray!20},
	x tick label style={rotate=90,anchor=east},
	x label style={at={(axis description cs:0.99,0)}},
	y label style={at={(axis description cs:0,1)},rotate=-90},
	scaled y ticks = false,
	/pgf/number format/.cd,fixed,precision=4,
	/pgf/number format/use comma          
    }   
\usetikzlibrary{backgrounds}	%um in den Hintergrund zeichnen zu können https://tex.stackexchange.com/questions/18200/draw-edges-and-paths-in-the-background-of-nodes-in-tikz
%%%%%%%%%%%%%%%%%%%%%%%%

\begin{document}
\hypersetup{CJKbookmarks=true}
\begin{sloppypar} 
%Hier beginnt das eigentliche Dokument. Ratsam ist es, dieses in Unterabschnitte aufzuteilen. Diese werden in seperaten Dokumenten (*.tex) bearbeitet und an dieser Stelle zum Gesamtdokument zusammengefügt. Um die Kompilierungszeit zu minimieren können Abschnitte, die gerade nicht bearbeitet werden einfach auskommentiert werden. Um eine übersicht über die Reihenfolge zu wahren ist es ratsam die Dateibezeichnung entsprechend zu nummereieren. In Texmaker werden links die einzelnen Abschnitte angezeigt.
%TIpp: aktiviere unter [Option]->[aktuelle Datei als 'Masterdatei' erklären]. Damit wird immer dieses Dokument kompiliert und alle oben eingebundenen Bibliotheken mit geladen. Egal in welchem Unterdokument gerade gearbeitet wird.
%\pagenumbering{Roman}					%Start der römischen Seitennummerierung

\begin{comment}
\headerOff								%Kopfzeile ausblenden bis zum Textteil (selbst definiertes Kommando)
\input{01Titelseite}
\input{02Aufgabenstellung}
\headerOn								%Kopfzeile beginnen (selbst definiertes Kommando)
\input{03Kurzreferat}
\tableofcontents						%Inhaltsverzeichnis. Wird automatisch erstellt
\newpage
\input{04AbkürzungSymbolVerzeichnis}
\pagenumbering{arabic}					%Start der arabischzen Seitennummerierung
\input{10MotivationUndProblemstellung}
\input{20StandDerForschungUndEntwicklung}
\input{30ZielsetzungUndInhaltDerArbeit}
%%%%%%%%%%%%%%%%%%%%%%%%%%%%%%%%%%%%%%%%%%%%%%%%%%%%%%%%%%%%%%%%%%%%%%%%%%%%%%%%%%%%%%%%%%%%
%%% |||||||||||                |||||||||||        |||||||||||                ||||||||||| %%%
%%% vvvvvvvvvvv Arbeitsbereich vvvvvvvvvvv %%%%%% vvvvvvvvvvv Arbeitsbereich vvvvvvvvvvv %%%

%Die entsprechenden Dateinamen ab hier entsprechend der Kapitelüberschrift benennen und die Nummerierung gegebenenfalls anpassen. Die Nummerierung muss dabei nicht zwangsweise den Kapitelüberschriften entsprechen, erhöht aber die Übersicht in der linken Editor-Spalte (Struktur) bzw. im Ordner.
%Achtung: nach dem Umbenennen kann es sein, dass Latex die zugeordneten Dateien nicht mehr findet und eine entsprechende Fehlermeldung ausgibt. Dann einfach via [Werkzeuge] -> [Aufräumen] -> Dateien löschen

\input{40Hauptteil}
\input{50Kapitelüberschrift}	
\input{55AbschnittMitVielUebersetzungsaufwand}
\input{60TabellenAbbildungenUndHinweiseLatex}
\input{70ZitierenUndQuellen}

%\input{...Weitere Kapitel oder Unterkapitel in seperaten Dateien}


%%% ^^^^^^^^^^^ Arbeitsbereich ^^^^^^^^^^^ %%%%%% ^^^^^^^^^^^ Arbeitsbereich ^^^^^^^^^^^ %%%
%%% |||||||||||                |||||||||||        |||||||||||                ||||||||||| %%%
%%%%%%%%%%%%%%%%%%%%%%%%%%%%%%%%%%%%%%%%%%%%%%%%%%%%%%%%%%%%%%%%%%%%%%%%%%%%%%%%%%%%%%%%%%%%

\input{9991ZusammenfassungUndAusblick}
\setlength{\emergencystretch}{3em}

%Quellenverzeichnis. Die entsprechende Datei wird in der Präambel der Masterdatei deklaiert (z. B. \addbibresource{9999Quellen.bib})
%\renewcommand{\refname}{Quellenverzeichnis}
%\printbibliography
\printbibliography[title=Literaturquellen,nottype=online]
\printbibliography[title=Onlinequellen,type=online]
\newpage

%Abbildungsverzeichnis. Wird automatisch erstellt
\phantomsection 
\addcontentsline{toc}{section}{Abbildungsverzeichnis}
\listoffigures
\newpage

%Tabellenverzeichni. Wird automatisch erstellt
\phantomsection 
\addcontentsline{toc}{section}{Tabellenverzeichnis}
\listoftables
\newpage

%Quellcodeverzeichnis. Wird automatisch erstllt
{\color{red}Diese und die folgenden drei Zeilen bitte auskommentieren, wenn kein Quellcode in der Arbeit eingebunden wurde:}
	\phantomsection
	\addcontentsline{toc}{section}{\lstlistlistingname}
	\lstlistoflistings
	\newpage
	
\cleardoubleemptypage					%Leere Seite einfügen wenn auf einer Forderseite beendet wurde	%Datei nicht bearbeiten. Eintragungen werden automatisch vorgenommen
\input{9993Selbständigkeitserklärung}	%<--Datei ergänzen
\headerOff
\input{9994Anhang}
\end{comment}


\headerOn
\thispagestyle{empty}					% keine Seitenzahl auf der Titelseite

\begin{figure}[!htb]
\vspace*{-1.0cm}
\begin{addmargin}{-0.1\linewidth}
\href{https://bildungsportal.sachsen.de/opal/auth/RepositoryEntry/1980137473}{\includegraphics[width=7cm]{Bilder/IFKM_Logo.png}}
\end{addmargin}
\end{figure}



\begin{center}
\vspace{4.5cm}
{\fontsize{28}{28} \selectfont{\textbf{\MakeUppercase{Messwertverarbeitung}}}}\\
\vspace{0.5cm}
{\fontsize{28}{28} \selectfont{\textbf{\MakeUppercase{und}}}}\\
\vspace{0.5cm}
{\fontsize{28}{28} \selectfont{\textbf{\MakeUppercase{Diagnosetechnik}}}}\\
\vspace{5.5cm}
    
\end{center}
\begin{flushleft}
\begin{tabular}{p{3.5cm} l}
Verantwortlicher: & Dr.-Ing. Zhirong Wang\\
\end{tabular}\\
\vspace{1.0cm}

\begin{tabular}{p{3.5cm} l}
\multicolumn{2}{l}{Professur für Dynamik und Mechanismentechnik}\\
\multicolumn{2}{l}{Institut für Festkörpermechanik}\\
\multicolumn{2}{l}{Fakultät Maschinenwesen}\\
PO2019: & MW-MB-SIM-12, MW-MB-KST-12, MW-MB-AKM-19\\
PO2012:: & MB-KS-10,  MB-SM-12\\
PO2006: &  MT14: Höhere Dynamik (Teilmodul Messwertverarbeitung/Diagnostik)\\
\end{tabular}
\vspace{1.0cm}


\begin{tabular}{p{3.5cm} l}
Datum der Abgabe: & \today\\			% Kann bei bekanntem Datum auch durch dieses ersetzt werden
\end{tabular}
\end{flushleft}
%\cleardoubleemptypage					% Leere Seite einfügen 
%\section{Einführung}
%\ifx\allfiles\undefined
\sethlcolor{yellow}
\subsection{\textbf{Aufgaben und Inhalt}}
Siehe \href{run:Bilder/ Referenzen/ Mitschrift-MWV-SS2021-Handout.pdf}{Messwertverarbeitung und Diagnosetechnik}.
\subsection{\textbf{Klassifikation der Signale {\itshape x(t)}}}
Einteilungskriterium:\par
%\hspace{1em} - nach der zeitlichen Quantisierung\par
%\hspace{2em} $\bullet$ Ein Signal x(t) ist \hl{zeit-kontinuierlich}, wenn die Amplitude zu jedem beliebigen Zeitpunkt veränderbar ist.\par
\hspace{1em} - Nach der zeitlichen Quantisierung\par
\begin{itemize}[topsep=2pt]
     \item
     Ein Signal \itshape x(t) ist \hl{zeit-kontinuierlich}, wenn die Amplitude zu jedem beliebigen Zeitpunkt veränderbar ist.
     \item
     Ein Signal \itshape x(t) ist \hl{zeit-diskret}, wenn die Amplitude nur zur diskreten Zeitpunkten veränderbar ist.
\end{itemize}

\hspace{1em} - Nach Quantisierung der Amplitude
\begin{itemize}[topsep=2pt]
     \item
     \hl{analoges Signal}: Signale, die kontinuierlich veränderliche Zeit und Amplituden haben.
     \item
     \hl{digitales Signal}: Signale, deren Amplitude nur auf endliche Anzahl diskreter Werte beschränkt, zeit-diskretes Signal.
\end{itemize}

\hspace{1em} - Reelle und komplexe Signale\par
\hspace{1em} - Nach der Reproduzierbarkeit\par
\begin{itemize}[topsep=2pt]
    \item 
    deterministische Signale
    \item
    zufällige Signale (stochastische Signale)
\end{itemize}

% Einfügen von "Klassifikation_Signale.pdf", die Bildformat kann *.jpg, *.png, *.pdf, *.eps, *.tiff usw.
\begin{figure}[H]
\centering
\includegraphics[width=105mm]{Bilder/Bilder_01/Klassifikation_Signale.pdf}
\caption{Klassifikation\ von\ Signalen}
\label{fig:env}
\end{figure}

%\clearpage
\hspace{1em} - Einteilung nach der Leistung/ Energie:\par

% Wrapfigure: 3 Zeile, links gelegt, Breite 12.5em
\begin{figure}[H]
    \centering
    \subfigbottomskip=10pt
    \subfigcapskip=-3pt
    \subfigure[Bsp.:Ohm'sches Gesetz]{
		\includegraphics[width= 50mm]{Bilder/Bilder_01/Widerstand.png}}
	\subfigure[Formeldarstellung]{
		\includegraphics[width=70mm]{Bilder/Bilder_01/Formel_1.png}}
	  \\
	\caption{Physikalische Analogie}
\end{figure}
\hspace{1em} Definition:\par
\hspace{2em} - \hl {Momentanleistung} eines Signals {\itshape x(t)}:\par

\begin{equation}
    P_{x} (t)=x^{2} (t) \cdots Autoleistung
\end{equation}
\begin{equation}
    P_{xy} (t)=x(t)\cdot y(t) \cdots Kreuzleistung
\end{equation}

\hspace{2em} - \hl {Die mittelere Leistung} eines Signals {\itshape x(t)}:\par
\begin{equation}
    \bar{P_{x}} =\lim_{T \to \infty} \frac{1}{2T} \int\limits_{-T}^{T} x^{2} (t)dt
\end{equation}

\hspace{2em} - \hl {Die Gesamtenergie} eines Signals {\itshape x(t)}:\par
\begin{equation}
    E_{x}=\int\limits_{-\infty }^{\infty} P_{x} (t)dt=\int\limits_{-\infty}^{\infty}x^{2} (t)dt
\end{equation}


\begin{itemize}[topsep=2pt]
     \item
     \hl{Leistungssignal}: Ein Signal {\itshape x(t)} ist ein Leistungssignal, wenn seine Energie bzw. sein Integral der Leistung über unendlichen Zeitbereich ebenfalls unendlich ist. \par
     \begin{figure}[h]
         \centering
         \subfigbottomskip=10pt
         \subfigcapskip=3pt
         \subfigure[x(t)=sin($\omega$ t)]{
             \includegraphics[width=0.3\linewidth]{Bilder/Bilder_01/Sinus_Funktion.png}
         }\hspace{2cm}
         \subfigure[Mathematische\ Beschreibung]{
             \includegraphics[width=0.3\linewidth]{Bilder/Bilder_01/Leistungssignal.png}
        }
     \caption{Leistungssignal}
     \end{figure}
     
     \item
     \hl{Energiesignal}: Ein Signal {\itshape x(t)} ist ein Energiesignal, wenn seine Gesamtenergie begrenzt ist: $E_{x}<\infty \to \bar{P_{x}}=0$\par
\end{itemize}
     
\begin{figure}[H]
    \centering
    \includegraphics[width=0.5\linewidth]{Bilder/Bilder_01/Sinus_abklingt.png}
    \caption{Abklingende Sinus-Funktion}
    \label{fig:Abklingende Sinus-Funktion}
\end{figure}

\hspace{2em} - Nach der Kausalität (kausales Signal):\par
\begin{equation}
    x(t)=\left\{
	\begin{aligned}
	0, \quad \mbox{für}\ x<0\\
	beliebig, \quad \mbox{für}\ x\ge 0
	\end{aligned}
	\right
	.
\end{equation}

\subsection{\textbf{Mathematische Grundlagen}}
\subsubsection{\textbf{Harmonische Funktion}}
\begin{figure}[h]
    \centering
    \subfigbottomskip=10pt
    \subfigcapskip=3pt
    \subfigure[Sinusfunktion]{
        \includegraphics[width=0.4\linewidth]{Bilder/Bilder_01/Math_Sinus.png}
    }
    \subfigure[Parameter]{
        \includegraphics[width=0.4\linewidth]{Bilder/Bilder_01/Param_Sinus.png}
    }
    \caption{Einfache harmonische Funktion}
    \label{fig:Einfache harmonische Funktion}
\end{figure}

\begin{figure}[H]
    \centering
    \subfigbottomskip=10pt
    \subfigcapskip=3pt
    \subfigure[sin($\omega$t)]{
        \includegraphics[width=0.3\linewidth]{Bilder/Bilder_01/sin(omegaxt).png}
    }
    \subfigure[sin($\omega$t-$\varphi$)]{
        \includegraphics[width=0.3\linewidth]{Bilder/Bilder_01/sin(omegaxt-phi).png}
    }
    \subfigure[sin($\omega$t+$\varphi$)]{
        \includegraphics[width=0.3\linewidth]{Bilder/Bilder_01/sin(omegaxt+phi).png}
    }
    \caption{Caption}
    \label{fig:my_label}
\end{figure}


\subsubsection{\textbf{Komplexe Zahlen}}

\begin{figure}[H]
    \centering
    \subfigbottomskip=10pt
    \subfigcapskip=3pt
    \subfigure[]{
        \includegraphics[align=c,width=0.4\linewidth]{Bilder/Bilder_01/reele_achse.png}
    }
    \subfigure[]{
        \includegraphics[align=c,width=0.5\linewidth]{Bilder/Bilder_01/komp_zahl.jpg}
    }
    \caption{Strahl für reelle sowie komplexe Zahlen}
    \label{fig:Komplexe Zahl}
\end{figure}

\begin{comment}
\begin{figure}[H]
    \centering
    \begin{minipage}[c]{.45\textwidth}
        \subfigure[Strahl für reele Zahlen]{
        \includegraphics[align=c, width=.9\linewidth]{Bilder/Bilder_01/reele_achse.png}}
    \end{minipage}
    \begin{minipage}[t]{.45\textwidth}
         \subfigure[Strahl für komplexe Zahlen]{
        \includegraphics[align=c, width=1\linewidth]{Bilder/Bilder_01/komp_zahl.jpg}
    }
    \end{minipage}
\end{figure}
\end{comment}


\begin{itemize}[topsep=2pt]
    \item 
    \underline{z} =a+jb, \quad j=$\sqrt{-1}$
    \item
    In Polarkoordinate:\par
    z=r \cdot $e^{j\varphi}=\sqrt{a^{2}+b^{2}} \cdot e^{j\varphi}$
    \item
    Trigonometrische Form:\par
    $z=\underbrace{r \cdot cos(\varphi)}_{\textcolor{blue}{a}}+j \cdot \underbrace{r \cdot sin(\varphi)}_{\textcolor{blue}{b}}=r \cdot (cos(\varphi)+j \cdot sin(\varphi))$
    \item[]
    \fbox{
        \parbox{0.95\textwidth}{
        \hl{Eulersche-Formel:}
        \begin{center}
             $e^{j\varphi}=cos\varphi+j \cdot sin\varphi$\par
             $cos\varphi=\frac{1}{2} \cdot (e^{j\varphi}+e^{-j\varphi}), \quad    sin\varphi=\frac{1}{2j} \cdot (e^{j\varphi}-e^{-j\varphi})$
        \end{center}
        }
    }
    \item
    Komplexe konjugierte Zahl:\par
    $\underline{z}=a+jb, \quad \underline{z^{*}}=a-jb$
    \item
    Rechenregel von Komplexer Zahl (Selbsstudium)
    \item
    Umwandlung zwischen verschiedenen Darstellungsformen der komplexen Zahlen:\par
    \begin{figure}[H]
        \centering
        \includegraphics[width=0.5\textwidth]{Bilder/Bilder_01/umwandlung_komp_zahl.png}
        \caption{Verschiedene Darstellungen komplexer Zahlen}
        \label{fig:Verschiedene Darstellungen komplexer Zahlen}
    \end{figure}
    \item[]
    \centerline{$\underline{z}=a+jb=r \cdot e^{j\varphi}$}
\end{itemize}


\subsubsection{\textbf{Komplexe harmonische Funktion}}
\begin{figure}[H]
    \centering
    \includegraphics[width=0.5\linewidth]{Bilder/Bilder_01/komp_harm_funk.png}
    \caption{Komplexe harmonische Funktion}
    \label{fig:Komplexe harmonische Funktion}
\end{figure}

\begin{equation}
    Ae^{j\omega t}=A \cdot cos(\omega t)+jA \cdot sin(\omega t)
\end{equation}

 \fbox{
        \parbox{0.95\textwidth}{
        \hl{Wiederholung: Eulersche-Formel:}
        \begin{center}
        $e^{j\varphi}=cos\varphi+j \cdot sin\varphi$\par
        $cos\varphi=\frac{1}{2} \cdot (e^{j\varphi}+e^{-j\varphi}), \quad sin\varphi=\frac{1}{2j} \cdot (e^{j\varphi}-e^{-j\varphi})$
        \end{center}
        }
    }



\subsubsection{\textbf{Überlagerung zwei Harmonischen gleicher Frequenz}}
\begin{equation}
    x(t)=x_{1}(t)+x_{2}(t)=A \cdot \cos(\omega_{0}t)+B \cdot \sin(\omega_{0}t)
\end{equation}

\fbox{
     \parbox{0.95\textwidth}{
        \hl{Additionstheorem aus Mathematik:}
        \begin{center}
        $sin(x+y)=sinx \cdot cosy+siny \cdot cosx$\\
        $cos(x+y)=cosx \cdot cosy-sinx \cdot siny$\\
        \end{center}
      }  
}

\begin{equation}
\begin{split}
     x(t)=& A \cdot \cos (\omega_{0}t)+B \cdot \sin(\omega_{0}t)\\
         =& \underbrace{\sqrt{A^{2}+B^{2}}}_{\textcolor{red}{C} }   \cdot \Bigl(\underbrace{\underbrace{\frac{A}{\sqrt{A^{2}+B^{2}}}}_{\textcolor{red}{\sin \psi} }}_{\textcolor{blue}{\cos \varphi} }   \cdot \cos(\omega_{0}t)+\underbrace{\underbrace{\frac{B}{\sqrt{A^{2}+B^{2}}}}_{\textcolor{red}{\cos\psi}}}_{\textcolor{blue}{\sin \varphi}}  \cdot \sin(\omega_{0}t )\Bigr)  \\
         =& \textcolor{red}{C \cdot \sin(\omega_{0}+\psi)}\\
         =& \textcolor{blue}{C \cdot \cos(\omega_{0}t+\varphi)}\\
     \psi=&\arctan (\dfrac{A}{B})\\
  \varphi=&\arctan (\dfrac{-B}{A})   
\end{split}
\end{equation}

Für die Bestimmung der Winkel $\psi$ bzw. $\varphi$ im Wertbereich $0-2\pi$ benutzt man die \hl{atan2} Funktion. In MatLab $atan2(y,x)$:\: \underline{$\psi$=atan2(A,B) \quad $\varphi$=atan2(-B,A)}.

\fbox{
     \parbox{0.95\textwidth}{
        \hl{Zusammenfassung:}\par
        $
        x(t)=A \cdot \cos(\omega_{0}t)+B \cdot \sin(\omega_{0}t)=C\sin(\omega_{0}+\varphi)=C \cdot \sin(\omega_{0}+\psi)\\
       \cdot C=\sqrt{A^{2}+B^{2}}\\
        \cos\varphi=\frac{A}{C}, \quad \sin\varphi=-\frac{B}{C}\\
        \sin\psi=\frac{A}{C}, \quad \cos\psi=\frac{B}{C}\\
        \varphi=atan2(-B, A), \quad \psi=atan2(A, B)
        $\\
        \textcolor{blue}{\hl{atan2 Funktion} liefert den Wertbereich ($-pi, pi$), Syntax in MatLab:\; atan2(y, x).}
        }
    }
    
\subsubsection{\textbf{Überlagerung zweier Harmonischen verschiedener Frequenzen}}

\begin{flalign*}
    & \  x_1=A \cdot \cos(\omega_1t)
    & \ & x_2=B \cdot \cos(\omega_2t)
    && \\
    & \ \omega_1=2\pi f_1
    & \ & \omega_2=2\pi f_2
    && \\
    & \ T_1=\frac{1}{f_1}
    & \ & T_2=\frac{1}{f_2}
\end{flalign*}

\begin{figure}[H]
    \centering
    \includegraphics[width=0.7\textwidth]{Bilder/Bilder_01/Ueber_zwei_versch_Freq.png}
    \caption{Überlagerung zweier Harmonischen verschiedener Frequenz}
    \label{fig:Ueber_zwei_versch_Freq}
\end{figure}

\fbox{
    \parbox{0.95\textwidth}{
    \begin{itemize}
        \item 
        \hl{Wichtig:} \\
        Die Summe von zwei Harmonischen verschiedener Frequenz ist \hl{nicht mehr harmonisch}.
        \item
        Die Summe zweier periodischen Funktion wird \hl{nur dann periodisch}, wenn \hl{das Frequenzverhältnis zweier Funktionen rational ist}.
        \item[]
        \hspace{4em}$\frac{\omega_2}{\omega_1}=\frac{f_2}{f_1}=\frac{T_1}{T_2}=\frac{m}{n}$, m, n - ganze Zahlen, teilerfremd
        \item
        Die Periodendauer der resultierenden Funktion {\itshape x(t)}:
        \item[]
        \hspace{4em}$T=kgV(T_1,T_2), \quad T=mT_2=nT_1, \quad f_0=ggT(f_1,f_2)$
        \item
        Wenn das Frequenzverhältnis nicht rational ist, dann ist die {\itshape x(t)} nicht periodisch.
        \item[]
        \hspace{4em}$\frac{T_1}{T_2}\approx \frac{m}{n}$, \quad für relativ große ganze Zahlen m,n $\rightarrow$ {\itshape x(t)} ist \hl{quasi-periodisch}. 
    \end{itemize}
    }
}

\textbf{Beispiel 1:}\\
{\itshape $x(t)=5\cos(4t)+4sin(10\pi t)$}\\
Frage: Ist {\itshape x(t)} periodisch?\\
\textcolor{blue}{Das Frequenzverhältnis $\dfrac{\omega_2}{\omega_1}=\dfrac{10\pi}{4}$} ist \textcolor{red}{irrational $\Rightarrow$ {\itshape x(t)} ist nicht periodisch.}

\textbf{Beispiel 2:}\\
{\itshape $y(t)=5\cos(10\pi t)+7sin(30\pi t+0.5)$}\\
Frage: Ist {\itshape y(t)} periodisch? Wenn ja, wie groß ist seine Periode?\\
\textcolor{blue}{Das Frequenzverhältnis $\dfrac{10\pi}{30\pi}=\dfrac{1}{3}$} ist \textcolor{red}{rational $\Rightarrow$ {\itshape y(t)} ist periodisch.} \\
$T_1=\dfrac{1}{5}, \quad T_2=\dfrac{2\pi}{\omega_2}=\dfrac{1}{15}, \quad T=kgV(T_1,T_2)=\dfrac{1}{5}$\\
$f_1=5, \quad f_2=15, \quad f=ggT(f_1,f_2)=5$\\



\subsubsection{\textbf{Spaltfunktion}}
Nicht normierte Form:  $si(t)=\dfrac{\sin t}{t}$\\
normierte Form:        $sinc(t)=\dfrac{\sin(\pi t)}{\pi t}$\\
\begin{equation}
    \mbox{Für t=0:}
    si(0)=\lim_{t \to 0} \dfrac{\sin t}{t}=\lim_{t \to 0}  \dfrac{\cos t}{1}=1 
\end{equation}

\begin{figure}[H]
    \centering
    \includegraphics[width=0.8\textwidth]{Bilder/Bilder_01/si(t)_Überlagerung.png}
    \caption{Entstehung der Spaltfunktion Si(t)}
    \label{fig:si(t)_Überlagerung}
\end{figure}

\fbox{
    \parbox{0.95\textwidth}{
    \hl{Eigenschaften:}
    \begin{itemize}
        \item [a)]
        $si(t)$ ist eine \uuline{gerade} Funktion. $si(t)=si(-t)$
        \item [b)]
        $si(0)=1$
        \item [c)]
        Nullstellen von $si(t)$: $\;$ $\pm \pi, \pm 2\pi, \pm 3\pi, \cdots$
    \end{itemize}
    }
}


%%% wrapfigure: Bild rechts gestellt, während Sätze andererseitig
%%% Param: [4] vertikal 4 Zeilen, {r} Bild recht liegend, {12em} Breite

\begin{comment}
\begin{itemize}
    \item 
    \begin{wrapfigure}{r}{6cm}
        \centering
        \includegraphics[width=1\linewidth]{Bilder/Bilder_01/gerade_Funktion.png}
        \caption{Gerade Funktion}
        \label{fig:gerade}
    \end{wrapfigure}
    Eine Funktion $x(t)$ ist gerade, wenn es gilt:
    $x(t)=x(-t)$ $\Rightarrow$ achsensymmetrisch
    \item 
    \begin{wrapfigure}{r}{6cm}
        \centering
        \includegraphics[width=1\linewidth]{Bilder/Bilder_01/ungerade_Funktion.png}
        \caption{Ungerade Funktion}
        \label{fig:ungerade}
    \end{wrapfigure}
    Eine Funktion $x(t)$ ist ungerade, wenn es gilt:
    $x(t)=-x(-t)$ $\Rightarrow$ punktsymmetrisch
\end{itemize}
\end{comment}

\vspace{1.5cm}
\fbox{
   \parbox{.95\textwidth}{
       \begin{minipage}{.45\linewidth}
       \hl{Definition:}
            \begin{itemize}
                \item Eine Funktion $x(t)$ ist gerade, wenn es gilt:$x(t)=x(-t)$ $\Rightarrow$ achsensymmetrisch
                \item Eine Funktion $x(t)$ ist ungerade, wenn es gilt:$x(t)=-x(-t)$ $\Rightarrow$ punktsymmetrisch
            \end{itemize}
        \end{minipage}\vspace{0.5cm}
        \begin{minipage}{.45\linewidth}
            \hspace{0.6cm}
            \includegraphics[scale=0.15]{Bilder/Bilder_01/gerade_Funktion.png}
            %\caption{Gerade Funktion}
            \label{fig:gerade}
            \hspace{0.8cm}
            \includegraphics[scale=0.15]{Bilder/Bilder_01/ungerade_Funktion.png}
            %\caption{Ungerade Funktion}
            \label{fig:ungerade}
        \end{minipage}
   }
}




\subsubsection{\textbf{DIRAC-Impuls}}

%%% \{minipage} ist nur innerhalb eines Containers verwendbar
\begin{figure}[H]
    \begin{minipage}{.45\linewidth}
        \centering
        \includegraphics[scale=0.2]{Bilder/Bilder_01/Dirac-Impuls.png}
    \end{minipage}
    \begin{minipage}{.45\linewidth}
            \begin{equation}
                r_\tau (t)=\left\{
                \begin{array}{cl}
                    \dfrac{1}{\tau}, &  \ \mbox{\itshape{für}} -\dfrac{\tau}{2}\leq t\leq \dfrac{\tau}{2} \\
                    0,  &  \ sonst \\
                \end{array} \right.
                \nonumber
            \end{equation}
        \end{minipage}
\end{figure}

Die DIRAC-Impuls ist definiert als:\\
\begin{equation}
    \delta (t)=\lim_{\tau \to 0}r_\tau (t) 
\end{equation}

\fbox{
    \hl{Definition:}\\
    \begin{minipage}{.45\linewidth}
        \begin{figure}[H]
            \centering
            \includegraphics[scale=0.5]{Bilder/Bilder_01/Dirac.png}
            \caption{Dirac-Impuls}
            \label{fig:dirac}
        \end{figure}
    \end{minipage}
    \begin{minipage}{.45\linewidth}
        \begin{equation}
            \delta = \left\{
            \begin{array}{cc}
                 0, & \ \mbox{für} t\neq 0 \\
                 \neq 0, & \ \mbox{für} t=0 \\
                 \end{array} \right.
                 \nonumber
        \end{equation}
    \end{minipage}
}

\vspace{0.5cm}
\hl{Eigenschaften:}
\begin{enumerate}
    \item 
    $\delta (t)$ ist eine gerade Funktion, $\delta (t)=\delta (-t)$
    \item
    $\int\limits_{-\infty}^{\infty} \delta(t)dt=1$
    \item
    \hspace{1.77cm} 
    \begin{figure}[H]
        \centering
        \begin{minipage}{.45\textwidth}
            \centering
            \includegraphics[scale=0.6]{Bilder/Bilder_01/Eigen_delta_01.png}
        \end{minipage}
        \begin{minipage}{.45\textwidth}
            \begin{equation}
                \begin{split}
                    x_s(t)&\approx \cdots r_{\Delta t} \cdot (t-k\Delta t) \cdot \Delta t \cdot x(k\Delta t)\cdots\\
                    &=\sum_{k=-\infty}^{\infty} r_{\Delta t} \cdot (t-\textcolor{blue}{\underbrace{k\Delta t}_\tau}) \cdot x(\textcolor{blue}{\underbrace{k\Delta t}_\tau}) \cdot \textcolor{blue}{\underbrace{\Delta t}_{d\tau}} \\
                    & k \cdot \Delta t \rightarrow \tau \\
                    & \Delta t \rightarrow d\tau 
                \end{split}
                \nonumber
            \end{equation}
        \end{minipage}
    \end{figure}
\end{enumerate}

Bei der Eigenschaft \textbf{(3)}, wenn $\Delta t$ von Null tendiert, gilt:
\begin{equation}
    \begin{split}
      x(t)=&\lim_{\Delta t \to 0}x_s(t)\\
          =&\lim_{\Delta t \to 0}\sum_{k=-\infty}^{\infty} r_{\Delta t} \cdot (t-k\Delta t) \cdot x(k\Delta t) \cdot \Delta t \\
          =&\int\limits_{-\infty}^{\infty}\delta (t-\tau)x(\tau)d\tau  
    \end{split}
    \nonumber
\end{equation}
Daraus erhält man die Gleichung:\\
\begin{equation}
    \boxed{x(t)=\int\limits_{-\infty}^{\infty}x(\tau) \cdot \delta (t-\tau)d\tau}
\end{equation}
\\Wobei, $\delta (t-\tau)$ ist der Dirac-Impuls, der als seine Verschiebung entlang der t-Achse nach rechts um eine Stecke von t angesehen werden kann.\\

Daraus ergeben sich zwei Sonderfälle der Gleichung \; (11).

\begin{equation}
    \begin{split}
        &x(t)\delta (t)=x(0)\delta (t)\\
        &\int\limits_{-\infty}^{\infty}x(t)\delta (t)dt=x(0)\\
        &\int\limits_{-\infty}^{\infty}x(t)\delta (t-t_0)dt=x(t_0)
    \end{split}
    \nonumber
\end{equation}


\subsubsection{Faltungsprodukt}

\begin{figure}[H]
    \centering
    \includegraphics[scale=0.65]{Bilder/Bilder_01/Faltungsprodukt.png}
    \caption{LTI-System}
    \label{fig:lti}
\end{figure}



\begin{figure}[H]
    \begin{minipage}{.45\linewidth}
        \centering
        \hspace{15 pt}
        \includegraphics[scale=0.7]{Bilder/Bilder_01/LTI-system.png}
        \nonumber
    \end{minipage}
    \begin{minipage}{.45\linewidth}
            \begin{equation}
               y(t)=T_r\{x(t)\}
            \nonumber
            \end{equation}
        \end{minipage}
\end{figure}

\begin{equation}
    \begin{split}
    y(t)=&T_r\left \{\int\limits_{-\infty }^{\infty }x(\tau ) \cdot \delta (t-\tau )d\tau \right \} \\
        =&\int\limits_{-\infty }^{\infty } T_r\left \{x(\tau ) \cdot \delta (t-\tau )d\tau \right \}\\
        =&\int\limits_{-\infty }^{\infty }x(\tau ) \underbrace{T_r\ \cdot {\delta (t-\tau )\}}_{\textcolor{blue}{Impulsantwort\; h(t),\; Gewichtfunktion}} d\tau 
\end{split}
\end{equation}

\fbox{
    \parbox{0.95\textwidth}{
        \hl{Faltungsprodukt:}
        \begin{equation}
            y(t)=\int\limits_{-\infty }^{\infty }x(\tau ) \cdot h(t-\tau )d\tau =x(t)\ast  h(t)
        \end{equation}
        Sonderfall:\\
        \hspace{1.77cm} Wenn $x(t)$ und $y(t)$ \uline{kausal} sind, d.h. sie sind für t<0 Null, denn gilt:\\
        \begin{equation}
            y(t)=x(t)\ast  h(t)=\int\limits_{0}^{\infty }x(\tau )h(t-\tau )d\tau
        \end{equation}
    }
}
\textbf{Eigenschaften:}
\begin{enumerate}
    \item 
    $x*h=h*x$
    \item
    $(x_1+x_2)*h=x_1*h+x_2*h$
    \item
    $(x*g)*h=x*(g*h)$
    \item
    \uuline{$x(t)*\delta(t-t_0)=x(t-t_0)$}
\end{enumerate}

\textbf{Grafische Darstellung des Faltungsvorgangs im Zeitbereich}:
\begin{figure}[H]
    \centering
    \begin{minipage}{.3\textwidth}
        \includegraphics[scale=0.3]{Bilder/Bilder_01/x0(t).png}
        \nonumber
    \end{minipage}
    \begin{minipage}{.03\textwidth}
            $\ast$
    \end{minipage}
    \begin{minipage}{.3\textwidth}
             \includegraphics[scale=0.3]{Bilder/Bilder_01/t0.png}
        \nonumber
    \end{minipage}
    \begin{minipage}{.03\textwidth}
            =
    \end{minipage}
    \begin{minipage}{.3\textwidth}
        \includegraphics[scale=0.3]{Bilder/Bilder_01/Faltungsprodukt_graphisch.png}
        \nonumber
    \end{minipage}
    \caption{Grafische Darstellung des Faltungsvorgangs}
    \label{fig:Faltung im Zeitbereich}
\end{figure}

Grafische Methode zur Berechnung der Faltung:\\
\begin{equation}
    y(t)=x(t)\ast h(t)=\int_{-\infty}^{\infty}x(\tau)h(t-\tau)d\tau
\end{equation}

\begin{figure}[H]
    \centering
    \begin{minipage}{.45\textwidth}
        \includegraphics[scale=0.5]{Bilder/Bilder_01/Faltung_grafisch.png}
    \end{minipage}
    \begin{minipage}{.45\textwidth}
        \begin{enumerate}
            \item[]
            \textbf{Schritte:}\\
            \item 
            Variable $\,$ t $\rightarrow$\ $\tau$. Skizze von x(t), $\tau(t)$.
            \item
            achsensymmetrische Spiegelung von $h(\tau) \rightarrow h(-\tau)$
            \item
            Verschiebung des gespiegelten Signals $\,$ $h(\tau)$ $\,$ von eimen Wert $t_1$\\
            $\Rightarrow$ $h(t_1-\tau)$
            \item
            Berechnung des Produktes der zwei Verläufe $x(\tau)h(t_1-\tau)$
            \item
            Bestimmung der Fläche unter dem Produkt der beiden Signalen, diese Fläche $\,$=$\,$ das Faltungsprodukt zum Zeitpunkt $t_1$.
            \item
            Wiederholung der Schritte \ding{194} bis \ding{196} für einen anderen t-Wert.
        \end{enumerate}
    \end{minipage}
    \caption{Schritte zur Berechnung einer Faltungsoperation}
\end{figure}

\begin{figure}[H]
    \centering
    \includegraphics[scale=0.4]{Bilder/Bilder_01/Zusammenfassung des Faltungsprozesses.png}
    \caption{Zusammenfassung des Faltungsprozesses}
    \label{fig:Zusammenfassung Faltung}
\end{figure}

\newpage
\section{Fourieranalyse periodischer kontinuierlicher Signale}
%\ifx\allfiles\undefined
\begin{figure}[H]
    \centering
    \begin{minipage}{.6\textwidth}
        \centering
        \includegraphics[scale=0.4]{Bilder/Bilder_02/harmonische.png}
        \caption{Harmonische Signale}
        \label{fig:har_sig}
    \end{minipage}
    \hspace{0.6cm}
    \begin{minipage}{.35\textwidth}
    \textbf{Merkmale}:
        \begin{itemize}
            \item 
            Amplitude: \, A
            \item
            Periodendauer: \,T
            \item
            Kreisfrequenz: \, $\omega=\dfrac{2\pi}{T}$
            \item
            Frequenz: \, $f=\dfrac{\omega}{2\pi}=\dfrac{1}{T}$
        \end{itemize}
    \end{minipage}
\end{figure}

\begin{figure}[H]
    \centering
    \begin{minipage}{.6\textwidth}
        \centering
        \includegraphics[scale=0.45]{Bilder/Bilder_02/periodische.png}
        \caption{Periodische Signale}
        \label{fig:Period_sig}
    \end{minipage}
    \hspace{0.6cm}
    \begin{minipage}{.35\textwidth}
        \textbf{Merkmale}:
            \begin{itemize}
                \item 
                Periodisch: \, $x(t)=x(t+T)$
                \item
                Periodendauer:\, $T_1$
                \item
                Grundfrequenz:\, $f_1=\dfrac{1}{T_1}$
                \item
                Grundkreisfrequenz:\, $\omega_1=2\pi f_1=\dfrac{2\pi}{T_1}$
            \end{itemize}
    \end{minipage}
\end{figure}

\begin{figure}[H]
    \centering
    \begin{minipage}{.45\textwidth}
        \begin{figure}[H]
            \centering
            \includegraphics[scale=0.4]{Bilder/Bilder_02/Überlagerung Harmonische_01.png}
            \nonumber
        \end{figure}
        \begin{equation}
        \begin{split}
               u_{s u m}(t)=&\cos \left(\omega_{0} t\right)+\frac{1}{3} \cos \left(3 \omega_{0} t+\pi\right) \\+&\frac{1}{5} \cos \left(5 \omega_{0} t\right)
        \end{split}
             \nonumber
        \end{equation}
    \end{minipage}
    \begin{minipage}{.45\textwidth}
        \centering
        \begin{figure}[H]
            \centering
            \includegraphics[scale=0.4]{Bilder/Bilder_02/Überlagerung Harmonische_02.png}
            \nonumber
        \end{figure}
        \begin{equation}
        \begin{split}
            u_{s u m}(t)=&\cos \left(\omega_{0} t\right)+\frac{1}{3} \cos \left(3 \omega_{0} t\right) \\ +&\frac{1}{5} \cos \left(5 \omega_{0} t\right)
        \end{split}
        \nonumber
        \end{equation}
    \end{minipage}
        \caption{Überlagerung mehrerer Harmonischen}
\end{figure}

\textbf{Spektrum und Fourier-Reihe}:\\
\begin{figure}[H]
    \centering
    \begin{minipage}{.5\textwidth}
        \centering
        \includegraphics[scale=0.6]{Bilder/Bilder_02/Lichtsspektren.png}
        \nonumber
    \end{minipage}
    \hspace{0.6cm}
    \begin{minipage}{.4\textwidth}
        \textit{\textbf{Spektrum:} Das Muster des monochromatischen Lichts, das in der Reihenfolge der Wellenlänge (oder Frequenz) gestreut wurde, wird als optisches Spektrum bezeichnet.}
    \end{minipage}
    \caption{Spektrum}
    \label{fig:Spektrum}
\end{figure}

\begin{figure}[H]
    \centering
    \subfigbottomskip=10pt
    \subfigcapskip=3pt
    \subfigure[Fourier:\, 1768-1830]{
        \includegraphics[scale=0.4]{Bilder/Bilder_02/Joseph Fourier.png}
    }
    \subfigure[Fourier-Reihe]{
        \includegraphics[scale=0.8]{Bilder/Bilder_02/FR.png}
        \nonumber
    }
    \caption{Fourier und FR}
    \label{fig:Fourier und FR}
\end{figure}


\subsection{\textbf{Fouirerreihe}}
\subsubsection{\textbf{Orthogonale Reihenentwicklung}}
Periodisches Signal $x(t)$ mit Periodendauer soll durch einen Satz von Basisfunktionen approximiert werden.\\
\textbf{Basisfunktion:} $g_0^{\textcolor{red}{(t)}},g_1^{\textcolor{red}{(t)}},g_2^{\textcolor{red}{(t)}},\cdots,g_k^{\textcolor{red}{(t)}},\cdots$

Die Approximation von $x(t)$:\\
\begin{equation}
    x_{\text {approx }}(t)=\sum_{k=0}^{\infty} a_{k} g_{k}(t) \quad \mbox {$a_k$-Koeffizient der Reihe}
\end{equation}

Kriterien zur Bestimmung der Koeffizienten $a_k$: $\rightarrow$ \textbf{bleibend Fehlerquadrat.}

\begin{equation}
\begin{aligned}
    Q F &=\int_{a}^{b}\left(x(t)-x_{\text {approx }}(t)\right)^{2} d t \leadsto \min \\
       &=\int_{a}^{b}\left(x(t)-\sum_{l=0}^{\infty} a_{l} g_{l}(t)\right)^{2} d t \leadsto \min
\end{aligned}
\end{equation}
Damit \textbf{QF} minimal wird, muss die Folgende Ableitung gelten:\\
\begin{equation}
\frac{\partial Q F}{\partial a_{k}}=0, \quad \mbox{\textit{\textbf{für}}} \; k=0,1,2, \ldots
\end{equation}

Die Differentiation:\\
\begin{equation}
\begin{split}
    \frac{\partial Q F}{\partial a_{k}}=\int_{a}^{b} 2\left[x(t)-\sum_{l=0}^{\infty} a_{l} g_{l}(t)\right]\left(-g_{k}\right) d t=0\\
    \int_{a}^{b}\left[x(t) g_{k}(t)-\sum_{l=0}^{\infty} a_{l} g_{l}(t) g_{k}(t)\right] \cdot d t=0
\end{split}
\end{equation}

\fbox{
    \parbox{1\linewidth}{
    Für jeden k, k=0,1,2,\ldots:
    \begin{center}
       \begin{equation}
       \setlength\abovedisplayskip{1pt} % shrink space
       \setlength\belowdisplayskip{1pt}
       \int_{a}^{b} x(t) \cdot g_{k}(t) \cdot d t=\sum_{l=0}^{\infty} \int_{a}^{b} a_{l} g_{l}(t) g_{k}(t) d t \eqno(*)
       \nonumber
       \end{equation}
    \end{center}
    }
}

Sonderfall:\\
Wenn die Basisfunktion $g_0^{\textcolor{red}{(t)}},g_1^{\textcolor{red}{(t)}},\cdots$ zueinander orthogonal sind, d.h.:\\
\begin{equation}
    \int_{a}^{b} g_{k}(t) \cdot g_{l}(t) d t= \begin{cases}0 & \text { für } k \neq l \\ K_{l} & \text { für } k=l\end{cases}
\end{equation}

Die Gleichung (*) vereinfacht sich zu:\\
\begin{equation}
    \int_{a}^{b} x(t) g_{k}(t) d t=a_{k} \cdot \int_{a}^{b} g_{k}(t) \cdot g_{k}(t) d t, \quad k=0,1,2, \cdots
\end{equation}

\begin{equation}
    \textcolor{blue}{\Rightarrow} a_{k}=\frac{\int_{a}^{b} x(t) g_{k}(t) d t}{\int_{a}^{b} g_{k}(t) \cdot g_{k}(t) d t}
\end{equation}


\subsubsection{\textbf{Reelle Fourierreihe (trigonometrische Fourierreihe)}}
Basisfunktion:\par
\begin{center}
    \begin{tabular}{c|c|c|c|c|c|c}
    \hline
    \centering
    1 & $\cos(\omega_1 t)$ & $\sin(\omega_1 t)$ & $\cos(2 \omega_1 t)$ & $\sin(2 \omega_1 t)$ & $\cos(3 \omega_1 t)$ & $\cdots$\\
    \hline
    $g_0$ & $g_{1c}$ & $g_{1s}$ & $g_{2c}$ & $g_{2s}$ & $g_{3c}$ & $\cdots$ \\
    $a_0$ & $a_1$ & $b_1$ & $a_2$ & $b_2$ & $a_3$ & $\cdots$ \\
    \hline
    \end{tabular}
\end{center}


Es lässt sich prüfen, dass sie zueinander orthogonal sind.\\
Für ein periodischer Verlauf \,$x(t)$, Periodendauer T, \,Grundfrequenz $f_1=\dfrac{1}{T}$, Grundkreisfrequenz \, $\omega_1=\dfrac{2\pi}{T}$.
\begin{equation}
\begin{aligned}
    x(t)=\underbrace{a_{0}}_{\textcolor{blue}{Mittelwert}} &+\underbrace{a_{1} \cdot \cos \omega_{1} t+b_{1} \cdot \sin \omega_{1} t }_{\textcolor{red}{Grundschwingung:\, 1.\, Harmonische}}\\
               &+\underbrace{a_{2} \cdot \cos 2\omega_{1} t+b_{2} \cdot \sin 2\omega_{1} t }_{\textcolor{green}{Grundschwingung: \,2.\, Harmonische}}\\
               &+\underbrace{a_{3} \cdot \cos 3\omega_{1} t+b_{3} \cdot \sin 3\omega_{1} t }_{\textcolor{purple}{Grundschwingung:\, 3.\, Harmonische}}\\
               &+\cdots
\end{aligned}
\nonumber
\end{equation}

Die obige Entwicklung is als folgende Gleichung zusammenzufassen:\\
\begin{equation}
\begin{aligned}
    &x(t)=a_{0}+\sum_{k=1}^{\infty}\left(a_{k} \cdot \cos k \omega_{1} t+b_{k} \cdot \sin k \omega_{1} t\right) \\
    &a_{0}=\underbrace{\frac{\int_{T} x(t) \cdot g_{0}(t) d t}{\int_{T} g_{0}(t) \cdot g_{0}(t) d t}}_{\textcolor{blue}{T}}=\frac{1}{T} \int_{T} x(t) d t \\
    &a_{k}=\frac{\int_{T} x(t) g_{k c}(t) d t}{\int_{T} g_{k c}(t) \cdot g_{k c}(t) d t}=\frac{\int_{T} x(t) \cdot \cos k \omega_{1} t d t}{\int_{T}\left[\cos \left(k \omega_{1} t\right)\right]^{2} \cdot d t}=\frac{2}{T} \int_{T} x(t) \cdot \cos k \omega_{1} t d t \\
    &b_{k}=\frac{\int_{T} x(t) g_{k s}(t) d t}{\int_{T} g_{k s}(t) \cdot g_{k s}(t) d t}=\frac{\int_{T} x(t) \cdot \sin k \omega_{1} t d t}{\int_{T}\left[\sin \left(k \omega_{1} t\right)\right]^{2} d t}=\frac{2}{T} \int_{T} x(t) \cdot \sin k \omega_{1} t d t
    \end{aligned}
   \label{con:Gl zur Berechnung FR}
\end{equation}

Zusammenfassung der trigonometrischer Fourier-Entwicklung und deren Variaten:\\
\begin{equation}
\begin{aligned}
    x(t) &=a_{0}+\sum_{k=1}^{\infty}\left(a_{k} \cdot \cos k \omega_{1} t+b_{k} \sin k \omega_{1} t\right) \\
    &=a_{0}+\sum_{k=1}^{\infty} c_{k} \cdot \cos \left(k \omega_{1} t+\varphi_{k}\right) \\
    &=a_{0}+\sum_{k=1}^{\infty} d_{k} \cdot \sin \left(k \omega_{n} t+\psi_{k}\right) \\
    c_{k} &=d_{k}=\sqrt{a_{k}^{2}+b_{k}^{2}}\\
    \varphi_{k} & =a \tan 2\left(-b_{k} , a_{k}\right) \\
    \psi_{k}&=\operatorname{atan} 2\left(a_{k}, b_{k}\right)\\
\end{aligned}
\end{equation}

Ergänzung: Der Wertebereich der Tangensfunktion:\\
\begin{figure}
    \centering
    \begin{minipage}{.45\linewidth}
        \centering
        \includegraphics{Bilder/Bilder_02/tangensfunktion.png}
        \nonumber
    \end{minipage}
    \hspace{1cm}
    \begin{minipage}{.45\linewidth}
        $\tan \alpha=\dfrac{y}{x}$\\
        $\alpha = \operatorname{atan2}(y,x)\in (-\pi,\pi)$
    \end{minipage}
    \caption{Tangensfunktion und Wertbereich}
\end{figure}

\begin{figure}
    \centering
    \begin{minipage}{.7\linewidth}
        \centering
        \includegraphics[scale=0.3]{Bilder/Bilder_02/Darstellung in beiden Bereichen.png}
        \caption{Darstellung eines periodischen Signals im Zeit- sowie Frequenzbereich}
    \end{minipage}
    \begin{minipage}{.25\linewidth}
        Ein periodisches Zeitsignal $x(t)$ besitzt \textbf{nur die Grundfrequenz und deren Vielfache.} Ein Beispiel siehe unten:
    \end{minipage}
\end{figure}

\begin{figure}[H]
    \centering
    \begin{minipage}{0.55\linewidth}
        \centering
        \includegraphics[scale=0.4]{Bilder/Bilder_02/Rechteck-Impuls.png}
        \nonumber
    \end{minipage}
    \begin{minipage}{0.4\linewidth}
        \begin{itemize}
            \item[-]
            $T=1s$
            \item[-]
            $f_1=\dfrac{1}{T}=1Hz$
            \item[-]
            $x(t)$ kann nur die diskreten Frequenzen $1Hz, 2Hz, 3Hz \ldots$ besitzen.
        \end{itemize}
    \end{minipage}
\end{figure}

Fourierreihe für gerade/\, ungerade periodische Signale:\\
\begin{itemize}
    \item 
    \textit{gerade Signale $x(t)=x(-t)$:}\\
    \begin{equation}
        \begin{aligned}
        a_0=&\frac{1}{T}\int_{-\frac{T}{2}}^{\frac{T}{2}}x(t)dt=\frac{2}{T}\int_{0}^{\frac{T}{2}}x(t)dt\\
        a_{k}=&\frac{2}{T} \int_{-T / 2}^{T / 2} x(t) \cos (k \omega_{1} t) d t
        =\frac{2}{T} \cdot 2 \cdot \int_{0}^{T / 2} \underbrace{x(t) \cdot \cos( k \omega_{1} t)}_{gerade}dt\\
        b_{k}=&\frac{2}{T} \int_{-T / 2}^{T / 2} \underbrace{x(t) \cdot \sin( k \omega_{1} t)}_{ungerade } d t=0
        \end{aligned}
    \end{equation}
    \item
        \textit{ungerade Signale $x(t)=-x(-t)$:}\\
            \begin{equation}
        \begin{aligned}
        a_{0}=&\frac{1}{T}\int_{-\frac{T}{2}}^{\frac{T}{2}}x(t)dt=0\\
        a_{k}=&\frac{2}{T} \int_{-T / 2}^{T / 2} x(t) \cos (k \omega_{1} t) d t \\
        b_{k}=&\frac{2}{T} \int_{T} x(t) \cdot \sin (k \omega_{1} t )d t=\frac{2}{T} \cdot 2 \cdot \int_{0}^{T / 2} x(t) \sin (k \omega_{1} t )d t
        \end{aligned}
\end{equation}
\end{itemize}

\textbf{Beispiel: gerade Rechteckimpuls:}\\
\begin{figure}[H]
    \centering
    \begin{minipage}{.45\linewidth}
        \centering
        \includegraphics[scale=0.24]{Bilder/Bilder_02/gerade_bsp.png}
        \nonumber
    \end{minipage}
    \begin{minipage}{.45\linewidth}
        \begin{itemize}
            \item
            Kraftverlauf $x(t)$ ist periodisch mit Periodendauer\, T
            \item
            Für den Kraftverlauf:
            \begin{itemize}
                \item [-]
                Periode \,T
                \item [-]
                Grundkreisfrequenz\, $\omega_1=\frac{2\pi}{T}$
                \item [-]
                nur diskrete Frequenz vorhanden $k\omega_1$
            \end{itemize}
            \item
            \textbf{Gesucht:\,Fourierreihe}
        \end{itemize}
    \end{minipage}
\end{figure}

Der gezeigte Rechteckimpuls ist in funktionaler Form umzuschreiben:\\
\begin{equation}
    x(t)= \begin{cases}0 & t \in(\dfrac{-T}{2},\dfrac{-\tau}{2}) \\ 
                       A & t \in(\dfrac{-\tau}{2},\dfrac{\tau}{2}) \\ 
                       0 & t \in(\dfrac{\tau}{2},\dfrac{T}{2})\end{cases}
    \nonumber
\end{equation}

Nach dem Gleichungssystem \ref{con:Gl zur Berechnung FR}:\\
\begin{equation}
\begin{aligned}
    a_{0} &=\frac{1}{T} \int_{-T / 2}^{T / 2} x(t) d t=\frac{1}{T} \int_{-\tau / 2}^{\tau / 2} A \cdot d t=\frac{A \tau}{T} \\
    a_{k} &=\frac{2}{T} \int_{-T / 2}^{T / 2} x(t) \cdot \cos (k \omega \omega_{1} t) d t=\frac{2}{T} \int_{-\tau / 2}^{\pi / 2} A \cdot \cos (k \omega_{1} t) d t \\
    &=\left.\frac{2 A}{T} \cdot \frac{\sin (k \omega_{1} t)}{k \omega_{1}}\right|_{-\tau / 2} ^{\tau / 2}=\frac{2 A \textcolor{blue}{\tau}}{T} \frac{{\textcolor {red}{\xout{2}}} \sin (\frac{k \omega_{1} \tau}{2})}{\frac{k \omega_{1} \textcolor{blue}{\tau}}{\textcolor{red}{2} }} \\
    &=\frac{2 A \tau}{T} \cdot \operatorname{si}\left(\frac{k \omega_{1} \tau}{2}\right) \\
    b_{k} &=\frac{2}{T} \int_{-T / 2}^{T / 2} x(t) \cdot \sin (k \omega_{1} t )d t=0, \text { weil } x(t) \text { gerade ist. }
    \end{aligned}
\end{equation}

Daraus ergibt sich die Fourierreihe:\\
\begin{equation}
    x(t)=a_{0}+\sum_{k=1}^{\infty} \underbrace{\frac{2 A \tau}{T} \operatorname{si}\left(\frac{k \omega_{1} \tau}{2}\right)}_{\textcolor{blue}{A_k}} \cdot \cos \left(k \omega_{1} t\right)
    \nonumber
\end{equation}
Da $\cos(\alpha \pm \pi)=-\cos \alpha$:\\
\begin{equation}
    A_{k} \cdot \cos \left(k \omega_{1} t\right)= \begin{cases}A_{k} \cdot \cos \left(k \omega_{1} t\right) & A_{k} \geqslant 0 \\ \left|A_{k}\right| \cdot \cos \left(k \omega_{1} t+\pi\right) & A_{k}<0\end{cases}
\end{equation}

Amplitudenspektrum:\\
\begin{equation}
    a_0, \quad A_{k}=\frac{2 A \tau}{T} \cdot \operatorname{si}\left(\frac{k \omega_{1} \tau}{2}\right) \text {, an Frequenzstelle}\, k \omega_{1}
    \nonumber
\end{equation}


Betragsspektrum:\\
\begin{equation}
    x(t)=\frac{A \tau}{T}+\sum_{k=1}^{\infty}\left|\frac{2 A \tau}{T} \operatorname{si}\left(\frac{k \omega_{1} \tau}{2}\right)\right| \cdot \cos \left(k \omega_{1} t+\varphi_{k}\right)
\end{equation}

Phasenspektrum:\\
\begin{equation}
    \varphi_k=\left\{
    \begin{array}{cl}
    0, &  \ \dfrac{2nT}{\tau}\leqslant k \leqslant \dfrac{(2n+1)T}{\tau} \\
    \pi,  &  \ \dfrac{(2n+1)T}{\tau} \leqslant k \leqslant \dfrac{(2n+2)T}{\tau} \\
    \end{array} \right.
\end{equation}

Grafische Darstellung des Amplituden- sowie Phasenspektrums:\\
\begin{figure}[H]
    \centering
    \includegraphics[scale=0.8]{Bilder/Bilder_02/Amp_Pha_grafisch.png}
    \caption{Grafische Darstellung des Amplituden- sowie Phasenspektrums}
    \label{fig:amp_pha_grafisch}
\end{figure}

Fourierreihe für periodisches Signal:
\begin{center}
    $x(t)=x(t+T)$\\
\end{center}

Grundkreisfrequenz:
\begin{center}
    $\omega_1=\dfrac{2\pi}{T}$\\
\end{center}

Fourierreihe:
    \begin{equation}
        \begin{aligned}
        x_{(t)} &=a_{0}+\sum_{k=1}^{\infty}\left(a_{k} \cos k \omega_{1} t+b_{k} \sin k \omega_{1} t\right) \\
        &=a_{0}+\sum_{k=1}^{\infty} A_{k} \cos \left(k \omega_{1} t+\varphi_{k}\right)
        \end{aligned}
\end{equation}
Damit:
\begin{equation}
    \begin{aligned}
        A_k&=A(k\omega_1)\ldots Amplitudenspektrum\\
        \varphi_{k}&=\varphi(k\omega_1)\ldots Phasenspektrum
    \end{aligned}
\end{equation}
    

    


\subsubsection{\textbf{Komplexe Fourierreihe}}
\fbox{
    \parbox{.95\textwidth}{
        \hl{Eulersche Formel:}
        \begin{equation}
            \begin{aligned}
                \cos \alpha&=\dfrac{1}{2}(e^{j \alpha}+e^{-j \alpha})\\
                \sin \alpha&=\dfrac{1}{2j}(e^{j \alpha}-e^{-j \alpha})\\
                           &=-\dfrac{j}{2}(e^{j \alpha}-e^{-j \alpha})
            \end{aligned}
            \label{con:Eulersche Formel}
        \end{equation}
    }
}
Einsetzen der Gleichungen \ref{con:Eulersche Formel} in die reelle Fourierreihe:\\
\begin{equation}
\begin{aligned}
    x(t) &=a_{0}+\sum_{k=1}^{\infty} a_{k} \cdot \frac{1}{2}\left(e^{j k \omega_{1} t}+e^{-j k \omega_{1} t}\right)-\sum_{k=1}^{\infty} b_{k} \cdot \frac{j}{2}\left(e^{j k \omega_{1} t}-e^{-j k \omega_{1} t}\right) \\
         &=a_{0}+\sum_{k=1}^{\infty}\left(\frac{a_{k}-j b_{k}}{2} e^{j k \omega_{1} t}\right)+\sum_{k=1}^{\infty} \frac{a_{k}+j b_{k}}{2} e^{-j k \omega_{1} t} \\
         &=\underbrace{a_{0}}_{\textcolor{blue}{C_0 \cdot e^{j_0}}} +\sum_{k=1}^{\infty} \underbrace{\frac{a_{k}-j b_{k}}{2}}_{\textcolor{blue}{C_k} } \cdot e^{j k \omega_{1} t}+\sum_{k=-1}^{-\infty} \underbrace{\frac{a_{-k}+j b_{-k}}{2} }_{\textcolor{blue}{C_{-k}} }e^{j k \omega_{1} t}
\end{aligned}
\end{equation}

\begin{equation}
    \begin{aligned}
    &a_{k}=\frac{2}{T} \int_{T} x(t) \cdot \cos \left(k \omega_{1} t\right) d t \, \Rightarrow \, a_{-k}=a_{k} \\
    &b_{k}=\frac{2}{T} \int_{T} x(t) \cdot \sin \left(k \omega_{1} t\right) d t \, \Rightarrow \, b_{-k}=-b_{k}
    \end{aligned}
    \nonumber
\end{equation}

Dann gilt:\\
\begin{equation}
    \begin{aligned}
        x(t)&=x(t)=\sum_{k=-\infty}^{\infty}c_k \cdot e^{jk \omega_1 t}\\
        c_{0} &=a_{0}=\frac{1}{T} \int_{T} x(t) d t \\
        c_{k} &=\frac{1}{2}\left[\frac{2}{T}\left(\int_{T} x(t) \cdot \cos (k \omega_{1} t) d t-j \cdot \int_{T} x(t) \cdot \sin (k \omega_{1} t) d t\right)\right] \\
        &=\frac{1}{T} \int_{T} x(t)[\underbrace{\cos (k \omega_{1} t)-j \cdot \sin( k \omega_{1} t)}_{\textcolor{blue}{e^{-j k \omega_{1} t}}}] d t \\
        &=\frac{1}{T} \int_{T} x(t) e^{-j k \omega_{1} t} d t
    \end{aligned}
    \nonumber
\end{equation}

Zusammenfassung:\\
\fbox{
    \parbox{.95\textwidth}{
    \hl{Komplexe Fourierreihe:}
        \begin{equation}
        \begin{aligned}
        &x(t)=x(t+T), \quad \omega_{1}=\frac{2 \pi}{T} \\
        &x(t)=\sum_{k=-\infty}^{\infty} c_{k} \cdot e^{j k \omega_{1} t}, \quad c_{k}=\frac{1}{T} \int_{T} x(t) e^{-j k \omega_{1} t} d t \\
        &k=0, \pm 1, \pm 2,\ldots
        \end{aligned}
\end{equation}
    }
}

$C_k$ ist eine komplexe Zahl,
\begin{figure}[H]
    \centering
    \begin{minipage}{.45\textwidth}
    \begin{center}
        \begin{equation}
            \begin{split}
                   c_{k}&=|c_k|\cdot e^{j\varphi k}\\
                   |c_k|&=\dfrac{1}{2} \sqrt{a_{k}^{2}+b_{k}^{2}}=\dfrac{1}{2} A_k\\
                   \varphi_k&=\operatorname{arg}(c_k)=\operatorname{atan2}(\operatorname{Im(c_k)},\operatorname{Re}(c_k))\\
                            &=\operatorname{atan2}(-b_k, \, a_k)\\
                    c_k&=c_{-k}^{*} \quad * \,-\, konjugiert \; komplex        
            \end{split}
            \nonumber
        \end{equation}
    \end{center}

    \end{minipage}
    \begin{minipage}{.45\textwidth}
        \centering
        \includegraphics[scale=0.17]{Bilder/Bilder_02/komp_Zahl.jpg}
        \nonumber
    \end{minipage}
\end{figure}

Darstellung von Amplituden- und Phasenspektrum:
\begin{itemize}
    \item
    Wegen $c_k=c_{-k}^{*}$ \, $\rightarrow$ \, $|c_k|=c_{-k}$, \, $\varphi_k=-\varphi_{-k}$\\
    $\rightsquigarrow$ \uline{Betragspektrum ist gerade, \, Phasenspektrum ist ungerade.}
    \begin{figure}[H]
        \centering
        \includegraphics[scale=0.08]{Bilder/Bilder_02/gerade_ungerade.jpg}
        \nonumber
    \end{figure}
    \item
    für gerade \textcolor{red}{reelle} Signale $x(t)=x(-t)$:
    \begin{equation}
        \begin{aligned}
            b_k&=0\\
            c_k&=\dfrac{a_k-jb_k}{2}=\dfrac{a_k}{2}
        \end{aligned}
    \end{equation}
    \uline{$c_k$ ist reell $\Longleftrightarrow$ $x(t)$ ist gerade.}
    \item
    für ungerade \textcolor{red}{reelle} Signale $x(t)=-x(-t)$:
    \begin{equation}
        \begin{aligned}
            a_k&=0\\
            c_k&=\dfrac{a_k-jb_k}{2}=-\dfrac{j b_k}{2}=\dfrac{b_k}{2j}
        \end{aligned}
    \end{equation}
    \uline{$c_k$ ist imaginär $\Longleftrightarrow$ $x(t)$ ist ungerade.}
\end{itemize}

\textbf{Beispiel:}
\begin{figure}[H]
    \centering
        \begin{minipage}{.45\textwidth}
            \centering
            \includegraphics[scale=0.1]{Bilder/Bilder_02/rechteck_impuls_bsp.png}
            \caption{Rechteck-Puls und sinc-Funktion}
            \label{fig: symmetrie}
            %\nonumber
        \end{minipage}
        \hspace{1cm}
        \begin{minipage}{.45\textwidth}
            \begin{enumerate}
                \item 
                $x(t)$ ist periodisch
                \item
                Periode T
            \end{enumerate}
        \end{minipage}
\end{figure}

\begin{equation}
\begin{aligned}
x(t) &=\sum_{k=-\infty}^{\infty} c_{k} \cdot e^{i k \omega_{1} t} \\
c_{k}&=\frac{1}{T} \int_{T} x(t) \cdot e^{-j k \omega_{1} t} d t\\
     &=\frac{1}{T} \int_{-\tau / 2}^{\tau / 2} A \cdot e^{-j k \omega_{1} t} d t\\
     &=\frac{A \tau }{T}\cdot \operatorname{Si} \left( \dfrac{k\pi\tau}{T} \right)= \frac{A \tau }{T}\cdot  \operatorname{Si}(\dfrac{1}{2} \cdot k\tau\omega_1 )  
\end{aligned}
\nonumber
\end{equation}

Plausibilität: \, \textbf{$x(t)$ ist gerade \,$\rightarrow$ \, $c_k$\, reell}\\
\begin{figure}[H]
    \centering
    \includegraphics[scale=0.4]{Bilder/Bilder_02/Ein_Zwei_seitig.jpg}
    \caption{Ein- und Zweiseitiges Amplitudenspektrum}
\end{figure}


\subsubsection{\textbf{Parsevalsche Gleichung}}
Die Durchschnittleistung eines reellen periodischen Signals:

\begin{equation}
    \begin{aligned}
        \overline{P_{x}} &=\frac{1}{T} \int_{T} x^{2}(t) d t \\
        &=\frac{1}{T} \int_{T}\left[a_{0}+\sum_{k=1}^{\infty} a_{k} \cdot \cos (k \omega_{1} t)+\sum_{k=1}^{\infty} b_{k} \sin (k \omega_{1} t)\right]^{2} d t \\
        &=\frac{1}{T}\left[T \cdot a_{0}^{2}+\frac{T}{2} \sum_{k=1}^{\infty} a_{k}^{2}+\frac{T}{2} \sum_{k=1}^{\infty} b_{k}^{2}\right] \\
        &=a_{0}^{2}+\frac{1}{2} \sum_{k=1}^{\infty}\left(a_{k}^{2}+b_{k}^{2}\right) \, \textcolor{blue}{\leftarrow  reelle \;Fourierreihe\,} \\
        \overline{P_{x}} &=\frac{1}{T} \int_{T}^{2} x^{2}(t) d t=\frac{1}{T} \int_{T}\left(\sum_{k=-\infty}^{\infty} a_{k} \cdot e^{j k \omega_{1} t}\right)^{2} d t \\
        &=\sum_{k=-\infty}^{\infty}\left|c_{k}\right|^{2}
    \end{aligned}
    \nonumber
\end{equation}

\fbox{
    \parbox{.95\textwidth}{
        \hl{Parsevalsche Gleichung:}
            \begin{equation}
                \overline{P_{x}}=a_{0}^{2}+\frac{1}{2} \sum_{k=1}^{\infty}\left(a_{k}^{2}+b_{k}^{2}\right)=\sum_{k=-\infty}^{\infty}\left|c_{k}\right|^{2}
            \end{equation}
        }
}

\subsubsection{\textbf{Gibbs Phänomen}}
\textbf{Für Rechteck-Puls:}
\begin{figure}[H]
    \centering
    \begin{minipage}[c]{.4\linewidth}
        \centering
        \includegraphics[width=1\linewidth]{Bilder/Bilder_02/Rechteck-Puls.png}
        \nonumber
    \end{minipage}
    \begin{minipage}[c]{.5\linewidth}
        \begin{equation}
            \begin{aligned}
                &x(t)=\dfrac{2A}{\pi}(\sin(\omega_1 t)+\dfrac{1}{3}\sin(3\omega_1 t) +\dfrac{1}{5}\sin(5\omega_1 t)+\ldots )\\
                &\omega_1 = \dfrac{2\pi }{T}  
            \end{aligned}
            \nonumber
        \end{equation}
    \end{minipage}
\end{figure}
Erst wird der Rechteck-Puls aus der Überlagerung mehrerer Harmonischen grafisch dargestellt: 

\begin{figure}[H]
    \centering
    \subfigure{
    \begin{minipage}[b]{0.45\linewidth}
        \includegraphics[width=.8\linewidth]{Bilder/Bilder_02/N=15.jpg}\vspace{4pt}
        \includegraphics[width=.8\linewidth]{Bilder/Bilder_02/N=51.jpg}
    \end{minipage}}
    \subfigure{
    \begin{minipage}[b]{0.45\linewidth}
        \includegraphics[width=.8\linewidth]{Bilder/Bilder_02/N=25.jpg}\vspace{4pt}
        \includegraphics[width=.8\linewidth]{Bilder/Bilder_02/N=101.jpg}
    \end{minipage}}
    \caption{Rechteck-Puls aus der Überlagerung mehrerer Harmonischen}
\end{figure}

Dann wird das "Gibbs-Phänomen" als folgende definiert und charakterisierst:

\begin{figure}[H]
    \centering
    \begin{minipage}[c]{.5\linewidth}
    \centering
        \includegraphics[width=.9\linewidth]{Bilder/Bilder_02/Fourier-Synthese.jpg}
        %\nonumber
    \end{minipage}
    \begin{minipage}[c]{.4\linewidth}
        \begin{itemize}
            \item
            "Über- und Unterschiessen"\, vor und nach einer Sprungstelle
            \item
            Höhe der größten überschwingende Welle beträgt gegen 9\% (8,94\%) der gesamten Sprunghöhe
            \item
            Selbst bei \infty \,vielen Harmonischen bleibt die größten überschwingende Welle ca. 9\%, \, \hl{klingt aber schnell ab.}
        \end{itemize}
    \end{minipage}
\end{figure}

Bestimmung der Anzahl von Harmonischen in Bezug auf die mittlere Leistung:
\begin{equation}
    \begin{aligned}
        P_m&=\sum_{k=-\infty}^{\infty }\left | C_k \right |^2=a_0^2+\frac{1}{2}\cdot \sum_{k=1}^{\infty}\left ( a_k^2+b_k^2\right)\\
        P_N&=\sum_{k=-N}^{N}\left | C_k \right |^2=a_0^2+\frac{1}{2}\cdot \sum_{k=1}^{\infty}\left ( a_k^2+b_k^2\right)  
    \end{aligned}
\end{equation}
Wenn $\dfrac{P_N}{P_m}=99\%$, dann hat die synthetisierte Profile $99\%$ Leistung des theoretischen Verlaufs.

\subsubsection{Eigenschaften der Fouierreihe}
\begin{enumerate}[label={(\arabic*)}]
    \item 
    Linearität:\\
    \begin{center}
            $\textit{FR}\{x(t)+y(t)\}=\textit{FR}\{x(t)\}+\textit{FR}\{y(t)\}$\\
    \end{center}
    \begin{figure}[H]
        \centering
        \includegraphics[width=0.95\linewidth]{Bilder/Bilder_02/Lineraritaet.jpg}
        %\caption{Caption}
        %\label{fig:my_label}
    \end{figure}
    \item
    Lineare Modifikation:\\
    \begin{equation}
    \begin{aligned}
        &\textit{FR} \{\dot{x}\}=\frac{d}{d t} \textit{FR}\{x(t)\}\\
        &\textit{FR}\left\{\int x(t) d t\right\}=\int \textit{FR}\{x(t)\} d t
    \end{aligned}
    \nonumber
    \end{equation}
    \item
    Verschiebungssatz:\\
    \begin{equation}
        \textit{FR}\left\{x\left(t-t_{0}\right)\right\}=\textit{FR}\{x(t)\} \cdot e^{-j k \omega_{1}t_0}
    \end{equation}
    \nonumber
    D.h.,durch die Verschiebung im Zeitbereich um einen Abstand $t_0$: $x(t\pm t_0)$ ist die Multiplikation mit $e^{\mp j k \omega_{1}t_0}$ im Frequenzbereich vorgesehen, und vice versa.
    \item
    Symmetrieeigenschaften:
        \begin{itemize}
            \item [-]
            gerade reelle Signale:\\
            $b_k=0, \quad c_0=a_0, \quad c_k=\frac{1}{2}a_k, \: k=1,2,\ldots$\\
            \textit{$c_k$ ist reell.}
            \item [-]
            ungerade Signale:\\
            $a_0=a_k=0, \quad c_k=\dfrac{b_k}{2j}$\\
            \textit{$c_k$ ist imagnär.}
        \end{itemize}
    Beispiel: s.o. Abb.\ref{fig: symmetrie}
\end{enumerate}

\subsection{Fouriertransformation und Analyse aperiodischer Signale}
\subsubsection{Von Fourierreihe zu Fouriertransformation}
\fbox{
    \parbox{.95\linewidth}{
        \textit{Wiederholung: Fourierreihe}
        \begin{equation}
            \begin{aligned}
                    C_n=&\dfrac{1}{T}\int_{-\frac{T}{2}}^{\frac{T}{2}}x(t)\cdot e^(-jn\omega_0 t)dt\\
                    \omega_0 -& Grundfrequenz, \: \omega_0 = \dfrac{2\pi}{T}\\
                    C_n -& Fourierreihe-Koeffizient, C_n = \left |C_n\right |\cdot e^(j\angle{C_n})\\
                    &Frequenzkomponente \: bei \:Frequenz n\cdot \omega_0
            \end{aligned}
            \nonumber
        \end{equation}
    }
}
Im folgenden wird der 












\begin{comment}
\textit{Was habe ich getan, aus welchem Grund und auf welche Art und Weise? Welche Ergebnisse sind daraus entstanden und welche Erkenntnisse/Schlussfolgerungen lassen sich daraus ableiten?}\\
\end{comment}



\newpage
%%%\setlength{\emergencystretch}{3em}

%Quellenverzeichnis. Die entsprechende Datei wird in der Präambel der Masterdatei deklaiert (z. B. \addbibresource{9999Quellen.bib})
%\renewcommand{\refname}{Quellenverzeichnis}
%\printbibliography
\printbibliography[title=Literaturquellen,nottype=online]
\printbibliography[title=Onlinequellen,type=online]
\newpage

%Abbildungsverzeichnis. Wird automatisch erstellt
\phantomsection 
\addcontentsline{toc}{section}{Abbildungsverzeichnis}
\listoffigures
\newpage

%Tabellenverzeichni. Wird automatisch erstellt
\phantomsection 
\addcontentsline{toc}{section}{Tabellenverzeichnis}
\listoftables
\newpage

%Quellcodeverzeichnis. Wird automatisch erstllt
{\color{red}Diese und die folgenden drei Zeilen bitte auskommentieren, wenn kein Quellcode in der Arbeit eingebunden wurde:}
	\phantomsection
	\addcontentsline{toc}{section}{\lstlistlistingname}
	\lstlistoflistings
	\newpage
	
\cleardoubleemptypage					%Leere Seite einfügen wenn auf einer Forderseite beendet wurde

\end{sloppypar}
\end{document}

%' Anführungszeichen'("Text", ``Text´´) durch \glqq{}Text\grqq{} ersetzen

%Bei allgemein bekannten Abkürzungen wie z. B., u. a. oder i. d. R. musst du nach jedem Einzelwort einen Punkt sowie ein Leerzeichen setzen, eine Ausnahme hiervon bildet usw. Hier kannst du am besten ein geschütztes Leerzeichen verwenden, sodass die Abkürzung nicht über zwei Zeilen verteilt steht.
%Im vollen Wortlaut gesprochene Abkürzungen schreibst du also mit einem Punkt, z. B. Mag. für Magister, Jh. für Jahrhundert und v. Chr. für vor Christi Geburt.
%Fällt das Ende der Abkürzung mit dem Satzende zusammen, setzt du nur einen Punkt, bei einem Frage- oder Ausrufesatz jedoch Punkt und Zeichen: Ist er wirklich schon Mag.? 